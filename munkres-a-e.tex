% munkres-a-e 
% !TEX encoding = UTF-8 Unicode
% https://docs.google.com/document/d/1ggGZML_7mP6YBbQ7vZybQTcZ91Eb7HESuhqzNPRQlw8/edit?tab=t.0
% https://www.site24x7.com/tools/time-stamp-converter.html 1737310380
% https://www.overleaf.com/read/kdwwjvqjrzwb#9fe3a6
% https://github.com/Pierre-Yves-Gaillard/About-Topology-by-Munkres
\documentclass[12pt,letterpaper]{article}
\usepackage{fancyhdr}
\fancyhf{}
\fancyfoot[R]{{\tiny munkres-a-e,\ \filemodprintdate{\jobname},\ \filemodprinttime{\jobname},\ 1737310380}} 
% https://latexonline.cc/ (doesn't work well; abandoned)
% (https://www.unixtimestamp.com/ bad) 
% LaTeX pense-bête https://docs.google.com/document/d/1kgwDp0rjOmd-9h5ryYvDeNTSUzUSdG13mo4ZNFToeYA/edit?tab=t.0
\renewcommand{\headrulewidth}{0pt}
\fancyfoot[C]{\thepage}
\usepackage[T1]{fontenc}
\usepackage[utf8]{inputenc}
\usepackage{amssymb,amsmath,amsthm} 
\usepackage[letterpaper,top=50pt,left=60pt,right=60pt,bottom=70pt]{geometry}
\usepackage{filemod}%https://stackoverflow.com/questions/2118972/latex-command-for-last-modified
\usepackage{microtype}
%\usepackage{comment}
\usepackage[pdfusetitle]{hyperref}%\texorpdfstring{$\boldsymbol{\tau_2}$}{tau2}
%\usepackage{marvosym}%\usepackage{showkeys}%\usepackage{showlabels}
\usepackage[shortlabels]{enumitem}
\pagestyle{fancy}
%\pagestyle{empty}
\setlength{\parskip}{5pt} % variable
\renewcommand{\baselinestretch}{1.1} % variable
\newtheorem{thm}{Theorem}%[section]
\newtheorem{cor}[thm]{Corollary}
\newtheorem{lem}[thm]{Lemma}
\newcommand{\B}{\mathcal B}
\newcommand{\eps}{\varepsilon}
\newcommand{\hs}{\bigskip\hrule\medskip}
\newcommand{\mc}{\mathcal}
\newcommand{\noi}{\noindent}%
\newcommand{\Q}{\mathbb Q}%
\newcommand{\sm}{\setminus}
\newcommand{\R}{\mathbb R}
\newcommand{\T}{\mathcal T}
\newcommand{\Z}{\mathbb Z}
\begin{document}% \tiny \scriptsize \footnotesize \small \normalsize \large \Large \LARGE \huge \Huge 
\begin{center}{\Huge A few comments about ``Topology'' by Munkres}\bigskip 

Pierre-Yves Gaillard\footnote{ORCID https://orcid.org/0000-0002-7960-1698} 
\end{center}%\centerline{Munkres}\bigskip % Some material was removed while passing from munkres-a-a to munkres-a-b. Link to munkres-a-a: https://docs.google.com/document/d/1AInZwQapnKWK62NshpdsJ7BJaCtnaj-JC1Wwl0cYaWA/edit?tab=t.0 - 1729512177

\tableofcontents\bigskip

\noi As the title indicates, we make a comments about the book \textbf{Topology} by James R. Munkres. This is a work in progress. The last version of this text is available here: \url{https://www.overleaf.com/read/kdwwjvqjrzwb#9fe3a6}. Another version is available here: \url{https://github.com/Pierre-Yves-Gaillard/About-Topology-by-Munkres}. 

\section{Chapter 1. Set Theory and Logic}

\noi$\bullet$ \textbf{Definition of $\R$ p. 31.} The object $\R$ is defined by assuming that there exists a set $\R$ having certain properties. We take this assumption for granted. Then it is easy to see that there are several sets having these properties. So, let $\R'$ be a set having the same properties as $\R$. Let $\Z_+',\Z'$ and $\Q'$ be to $\R'$ what $\Z_+,\Z$ and $\Q$ are to $\R$. 

\begin{thm}\label{Tumf}
There is a unique morphism of fields from $f:\R\to\R'$. This morphism is an isomorphism of ordered fields, and it induces isomorphisms $\Z_+\to\Z_+',\Z\to\Z'$ and $\Q\to\Q'$. 
\end{thm}

\begin{lem}\label{Lgh}
There is a unique map $g:\Z_+\to\Z_+'$ such that $g(0)=0$ and $g(n+1)=g(n)+1$ for all $n$ in $\Z_+$. Similarly, there is a unique map $h:\Z_+'\to\Z_+$ such that $h(0)=0$ and $h(n+1)=h(n)+1$ for all $n$ in $\Z_+'$. 
\end{lem}

\begin{proof} 
For $i\in\Z_+$ and $\varphi:\{1,\ldots,i\}\to\Z_+'$ define $\rho(\varphi)\in\Z_+'$ by $\rho(\varphi):=\varphi(i)+1$. Then the first statement follows from the Principle of Recursive Definition (Theorem~\ref{TPRD} p.~\pageref{TPRD}). The proof of the second statement is similar. 
\end{proof}

\begin{proof}[Proof of Theorem \ref{Tumf}] 
In the notation of Lemma \ref{Lgh}, set $u:=h\circ g$. Then $u:\Z_+\to\Z_+$ satisfies $u(0)=0$ and $u(n+1)=u(n)+1$ for all $n$ in $\Z_+$. One can easily prove that $u(n)=n$ by induction. The same argument works for $g\circ h$. This shows that $g:\Z_+\to\Z_+'$ and $h:\Z_+'\to\Z_+$ are inverse isomorphisms. Then we extend $g$ to morphisms $\Z\to\Z'$ and $\Q\to\Q'$, and similarly for $h$, and, arguing as before, we show that these morphisms are isomorphisms. More precisely, we see that there is a unique morphism $\Z\to\Z'$ extending $g$, and that this morphism is an isomorphism, and similarly for the morphism $\Q\to\Q'$. So we can make the identifications $\Z_+=\Z_+',\Z=\Z',\Q=\Q'$. To show that there is a unique morphism of fields $\R\to\R'$, and that this morphism is an isomorphism (inducing the identity of $\Q$), we argue as in Section \emph{Appendix to Chapter~1} in \emph{A few comments about ``Principles of Mathematical Analysis'' by Rudin}, available at \url{https://zenodo.org/records/13955297}. 
\end{proof} \smallskip

\hrule\medskip

\noi\textbf{Theorem 7.8. p. 50 of the book.} Recall the statement: 

\begin{thm}[Theorem 7.8. p. 50 of the book]
Let $A$ be a set. There is no injective map $f : \mathcal{P}(A) \to A$, and there is no surjective map $g : A \to \mathcal{P}(A)$.
\end{thm} 

Here is my favourite way of phrasing the argument showing that there is no surjective map $g : A \to \mathcal{P}(A)$. Let $g : A \to \mathcal{P}(A)$ be a map, and set $B:=\{a\mid a\notin g(a)\}$, so that we have, for all $a$ in $A$, 
$$
a\in B\iff a\notin g(a).
$$ 
Let $a_0$ be in $A$. If we had $g(a_0)=B$, we would get, for all $a$ in $A$, 
$$
a\in g(a_0)\iff a\notin g(a),
$$ 
and we immediately that setting $a:=a_0$ yields a contradiction. This shows that $B$ is not in the range of $g$. 

\hs

\noi$\bullet$ \textbf{Exercise 7.6. p. 51 of the book.} We say that two sets $A$ and $B$ \textbf{have the same cardinality} if there is a bijection of $A$ with $B$.

\begin{enumerate}
    \item[(a)] Show that if $B \subset A$ and if there is an injection
    \[
    f: A \to B,
    \]
    then $A$ and $B$ have the same cardinality. [Hint: Define $A_1 = A, B_1 = B$, and for $n > 1$, $A_n = f(A_{n-1})$ and $B_n = f(B_{n-1})$. (Recursive definition again!) Note that $A_1 \supset B_1 \supset A_2 \supset B_2 \supset A_3 \supset \cdots$ Define a bijection $h: A \to B$ by the rule
    \[
    h(x) = \begin{cases} 
    f(x) & \text{if } x \in A_n\sm B_n \text{ for some } n, \\
    x & \text{otherwise}.]
    \end{cases}
    \]
    \item[(b)] \textit{Theorem (Schroeder-Bernstein theorem).} If there are injections $f: A \to C$ and $g: C \to A$, then $A$ and $C$ have the same cardinality.
\end{enumerate} 

\noi\textbf{Solution.} (a) We will freely use the following two obvious facts: 

(F1) For $x\in A$ and $n\in\Z_+$ we have 
$$
x\in A_n\iff f(x)\in A_{n+1}\text{ and }x\in B_n\iff f(x)\in B_{n+1}.
$$ 

(F2) We have $\bigcap_{n\ge1}A_n=\bigcap_{\ge1}B_n=:I$. 

\noi Setting $A_n':=A_n\sm B_n,B_n':=B_n\sm A_{n+1}$, we get 
$$
A=\left(\bigcup_{n\ge1}A_n'\right)\cup\left(\bigcup_{n\ge1}B_n'\right)\cup I,
$$ 
and this union is disjoint. We also have 
$$
B=\left(\bigcup_{n\ge2}A_n'\right)\cup\left(\bigcup_{n\ge1}B_n'\right)\cup I.
$$ 
The injection $f$ induces bijections $f_n:A_n'\to A_{n+1}'$ (here we are using (F1)). To define a bijection $h:A\to B$, it suffices to define three bijections 
$$
u:\bigcup_{n\ge1}A_n'\to\bigcup_{n\ge2}A_n',\quad v:\bigcup_{n\ge1}B_n'\to\bigcup_{n\ge1}B_n',\quad w:I\to I.
$$ 
We define $u$ by $u(x)=f_n(x)$ if $x\in A_n'$, and take $v$ and $w$ to be the identity maps. 

\noi(b) We set $B:=g(C)\subset A$ and define $f':A\to B$ by $f'(a):=g(f(a))$. Then $f':A\to B$ satisfies the assumptions for $f:A\to B$ in (a). 

\hs 

\noi$\bullet$ \textbf{Exercise 8.7. p. 56 of the book.} Prove Theorem 8.4 p. 54. 

\noi\textbf{Solution.} Recall the statement of Theorem~8.4. 

\begin{thm}[Principle of Recursive Definition, Theorem 8.4 of the book]\label{TPRD}
Let $A$ be a set; let $a_0$ be an element of $A$. Suppose $\rho$ is a function that assigns, to each function $f$ mapping a nonempty section of the positive integers into $A$, an element of $A$. Then there exists a unique function
\[h: \mathbb{Z}^+ \to A\]
such that
\[
\begin{aligned}
h(1) &= a_0,\\
h(i) &= \rho(h|{\{1,\ldots,i-1\}}) \text{ for } i > 1.
\end{aligned}
\tag{$*$}
\]
\end{thm}

The formula $(*)$ is called a recursion formula for $h$. It specifies $h(1)$, and it expresses the value of $h$ at $i > 1$ in terms of the values of $h$ for positive integers less than $i$. 

The book gives a detailed proof of the particular case when $\rho(h|{\{1,\ldots,i-1\}})$ is equal to $\min(C\sm h(\{1,\ldots,i-1\}))$, where ``$\min$'' means ``\emph{minimum}'', and $C$ is an infinite set. A close inspection of this proof reveals that the sole property of the element $c$ of $C$ defined by the equality $c:=\min(C\sm h(\{1,\ldots,i-1\}))$ is that it depends only on the restriction $h|{\{1,\ldots,i-1\}}$. This implies that, if, in the proof given by the book, we replace ``$\min(C\sm h(\{1,\ldots,i-1\}))$'' with ``$\rho(h|{\{1,\ldots,i-1\}})$'', then we obtain a proof of Theorem~\ref{TPRD}. 

\hs

\noi$\bullet$ \textbf{Exercise 10.7 p. 67.} Let $J$ be a well-ordered set. A subset $J_0$ of $J$ is said to be \textbf{inductive} if for every $\alpha \in J$,
\[(S_\alpha \subset J_0) \Rightarrow \alpha \in J_0.\] 

\begin{thm}[The Principle of Transfinite Induction]\label{TPTI1}
If $J$ is a well-ordered set and $J_0$ is an inductive subset of $J$, then $J_0 = J$. 
\end{thm} 

\noi\textbf{Solution.} If $J_0\ne J$, let $\alpha$ be the least element of $J\sm J_0$. We get $S_\alpha \subset J_0$, and thus $\alpha \in J_0$, contradiction. 

\hs

\noi$\bullet$ \textbf{Exercise 10.10 p. 67 of the book.} Prove the following Theorem:  

\noi\textbf{Theorem.} Let $J$ and $C$ be well-ordered sets; assume that there is no surjective function mapping a section of $J$ onto $C$. Then there exists a unique function $h: J \to C$ satisfying the equation
\[h(x) = \min(C\sm h(S_x)) \tag{$\ast$}
\]
for each $x \in J$, where $S_x$ is the section of $J$ by $x$. 

\noi\textbf{Solution.}
\begin{enumerate}
\item[(a)] If $h$ and $k$ map sections of $J$, or all of $J$, into $C$ and satisfy ($\ast$) for all $x$ in their respective domains, show that $h(x) = k(x)$ for all $x$ in both domains.

\item[(b)] If there exists a function $h: S_\alpha \to C$ satisfying ($\ast$), show that there exists a function $k: S_\alpha \cup \{\alpha\} \to C$ satisfying ($\ast$).

\item[(c)] If $K \subset J$ and for all $\alpha \in K$ there exists a function $h_\alpha: S_\alpha \to C$ satisfying ($\ast$), show that there exists a function
\[k: \bigcup_{\alpha \in K} S_\alpha \to C\]
satisfying ($\ast$).

\item[(d)] Show by transfinite induction that for every $\beta \in J$, there exists a function $h_\beta: S_\beta \to C$ satisfying ($\ast$). [Hint: If $\beta$ has an immediate predecessor $\alpha$, then $S_\beta = S_\alpha \cup \{\alpha\}$. If not, $S_\beta$ is the union of all $S_\alpha$ with $\alpha < \beta$.]

\item[(e)] Prove the theorem.
\end{enumerate} 

\noi\textbf{Solution.} 

\noi(a) Otherwise there would be a least $x$ such that $h(x)\ne k(x)$, we would get $h(S_x)=k(S_x)$, and $(*)$ would yield a contradiction. 

\noi(b) We define $k$ by $k(x)=h(x)$ if $x<\alpha$ and $k(x)=\min(C\sm h(S_x))$ if $x=\alpha$, and verify that $k$ satisfies $(*)$. 

\noi(c) Set $k(x)=h_\alpha(x)$ if $x\in S_\alpha$. To show that $k(x)$ is well defined, we must check that $\beta>\alpha$ implies $h_\beta(x)=h_\alpha(x)$. But this follows from (a). 

\noi(d) Let $I$ be the set of all $\beta\in J$ such that there is a map $h_\beta:S_\beta\to C$ satisfying $(*)$. It suffices to show that $I$ is inductive. So, assume that $\beta$ is in $J$ and that $S_\beta\subset I$. We must show $\beta\in I$. To do that, we use (b) if $\beta$ has an immediate predecessor, and we use (c) if not. 

\noi(e) We define $h$ by 
$$
h(x)=
\begin{cases}
\min(C\sm h_x(S_x))&\text{if }x=\max(J)\\ 
h_{x+1}(x)&\text{if }x\ne\max(J),
\end{cases} 
$$ 
where ``$x\ne\max(J)$'' means ``$x\ne\max(J)$ if $J$ has a maximum'', and $x+1$ is the least element greater than $x$. Let us show that $h$ satisfies $(*)$, that is, $h(x)=\min(C\sm h(S_x))$. We can assume $x\ne\max(J)$ (in the above sense). We must show $h_{x+1}(x)=\min(C\sm h(S_x))$. Since we have $h_{x+1}(x)=\min(C\sm h_{x+1}(S_x))$ by (d) it suffices to prove $h(S_x)=h_{x+1}(S_x)$. Let $y$ be in $S_x$, that is, $y\in J$ and $y<x$. It is enough to verify $h(y)=h_{x+1}(y)$, that is, $h_{y+1}(y)=h_{x+1}(y)$. We have $y+1<x+1$, and thus $S_{y+1}\subset S_{x+1}$, and (a) implies $h_{x+1}|S_{y+1}=h_{y+1}$. This proves $h_{y+1}(y)=h_{x+1}(y)$, which is what we wanted. 

\hs

\noi$\bullet$ \textbf{Supplementary Exercise 11.1 p. 72 of the book.} 

\begin{thm}[General principle of recursive definition]\label{TGPRD1}
Let $J$ be a well-ordered set; let $C$ be a set. Let $\mc F$ be the set of all functions mapping sections of $J$ into $C$.
Given a function $\rho:\mc F\to C$, there exists a unique function $h:J\to C$ such
that $h(\alpha)=\rho(h|S_\alpha)$ for each $\alpha\in J$. 
\end{thm} 

\noi[Hint: Follow the pattern outlined in Exercise 10 of §10.] 

\noi\textbf{Solution.} A close inspection of the solution to Exercise 10 of §10 reveals that the sole property of the element $c$ of $C$ defined by the equality $c:=\min(C\sm h(S_x))$ is that it depends only on the restriction $h|S_x$. This implies that, if, in the proof given by the book, we replace ``$\min(C\sm h(S_x))$'' with ``$\rho(h|S_x)$'', then we obtain a proof of Theorem~\ref{TGPRD1}. 

\hs

\noi Here is a slightly different way of proving the General Principle of Recursive Definition. We state and prove Theorem~\ref{TGPTI} below, which we call General Principle of Transfinite Induction, and which generalizes both the usual Principle of Transfinite Induction (Exercise 10.7 p.~67 of the book and Theorem~\ref{TPTI1} p.~\pageref{TPTI1}. above) and the General Principle of Recursive Definition (Supplementary Exercise 1 p.~72 of the book and Theorem~\ref{TGPRD1} above). 

For each ordered set $X$ and each $x$ in $X$ we denote the subset $\{y\in X\ |\ y<x\}$ by $X(x)$. (This is the so-called \emph{section by} $x$.) Let $X$ be a well-ordered set, let $A$ be a set, let 
$$
\rho:\bigcup_{x\in X}A^{X(x)}\to A,
$$ 
where $A^{X(x)}$ stands for the set of all maps from $X(x)$ to $A$. (Note that the sets $A^{X(x)}$ with $x$ in $X$ are disjoint.) 

\begin{thm}[General Principle of Recursive Definition]\label{TGPRD2}
There is a unique map $f:X\to A$ such that $f(x)=\rho(f|X(x))$ for all $x$ in $X$. 
\end{thm} 

The main ingredient to prove Theorem \ref{TGPRD2} is 

\begin{thm}[General Principle of Transfinite Induction]\label{TGPTI}
If $P(X)$ is a property that a well-ordered set $X$ may or may not have, and if $P(X)$ holds whenever $P(X(x))$ holds for all $x$ in $X$, then $P(X)$ holds for all well-ordered set $X$. 
\end{thm} 

Before proving Theorem \ref{TGPTI} recall the usual Principle of Transfinite Induction: 

\begin{thm}[Principle of Transfinite Induction]\label{TPTI2} 
Let $U$ be a well-ordered set. If $Q(u)$ is a property that an element $u$ of $U$ may or may not have, and if $Q(u)$ holds whenever $Q(v)$ holds for all $v<u$, then $Q(u)$ holds for all $u$ in $U$. 
\end{thm} 

This is Exercise 10.7 p. 67 of the book. 

\begin{proof}[Proof of Theorem \ref{TGPTI}] 
Let $X$ be a well-ordered set. We assume that $P(X)$ holds whenever $P(X(x))$ holds for all $x$ in $X$, and we want to prove $P(X)$. Let $\mc D$ be the set of all downward closed subsets of $X$. Then we have $\mc D=\{X(x)\ |\ x\in X\}\cup\{X\}$, and $X(x)\subsetneqq X(y)$ if and only if $x<y$, and $X(x)\subsetneqq X$ for all $x$, and $\mc D$ is well-ordered by proper inclusion. We want to apply the Principle of Transfinite Induction (Theorem~\ref{TPTI2}). To this end we set $U:=\mc D$ and, for $D\in\mc D$ we define $Q(D)$ as being $P(D)$. Then Theorem~\ref{TPTI2} tells us that $P(D)$ holds for all $D$ in $\mc D$, and thus in particular for $D=X\in\mc D$. 
\end{proof} 

\begin{proof}[Proof of Theorem \ref{TGPRD2}] 
We denote the statement of Theorem \ref{TGPRD2} by $P(X)$, and we want to apply Theorem~\ref{TGPTI}. So we assume that $P(X(x))$ holds for all $x$ in $X$. It suffices to prove $P(X)$. 

Case 1: $X$ has a largest element $\infty\in X$. By assumption, for all $x<\infty$ there is a unique map $f_x:X(x)\to A$ such that $f(y)=\rho(f|X(y))$ for all $y<x$. It is easy to check that each $f_x$ is the restriction of $f_\infty$ to $X(x)$, and that, if we define $f:X\to A$ by $f(\infty):=\rho(f_\infty)$ and $f(x):=f_\infty(x)$ if $x<\infty$, then $f$ is the unique solution to our problem. 

Case 2: $X=\bigcup_{x\in X}X(x)$. We have maps $f_x:X(x)\to A$ as above, and it is easy to check that that map $f:X\to A$ defined by $f(x):=f_{x+1}(x)$, where $x+1$ is the successor of $x$, is the unique solution to our problem. 
\end{proof} 

To see why the General Principle of Transfinite Induction (Theorem~\ref{TGPTI}) generalizes the Principle of Transfinite Induction (Theorem~\ref{TPTI2}), note that we can define $P(X)$ in terms of the $Q(u)$ by decreeing the $P(X)$ holds if and only if 
$$
\Big((\exists\ u\in U)\ \big(X=U(u)\big)\Big)\implies Q(u).
$$ \hrule\medskip

\noi$\bullet$ \textbf{Supplementary Exercise 11.2 p. 72 of the book.} 

\noi(a) Let $J$ and $E$ be well-ordered sets; let $h: J \to E$. Show the following two statements are equivalent:
\begin{enumerate}
    \item[(i)] $h$ is order preserving and its image is $E$ or a section of $E$.
    \item[(ii)] $h(\alpha) = \text{smallest }[E - h(S_\alpha)]$ for all $\alpha$.
\end{enumerate}
[Hint: Show that each of these conditions implies that $h(S_\alpha)$ is a section of $E$; conclude that it must be the section by $h(\alpha)$.]

\noindent (b) If $E$ is a well-ordered set, show that no section of $E$ has the order type of $E$, nor do two different sections of $E$ have the same order type. [Hint: Given $J$, there is at most one order-preserving map of $J$ into $E$ whose image is $E$ or a section of $E$.] 

\noi\textbf{Solution.} 

\noi(a) For all $X\subset E$ set $X^c:=E\sm X$. For the sake of prudence, we change (ii) to: 

\noi(ii') $h(S_x)\ne E$ and $h(x)=\min(h(S_x)^c)$ for all $x$. 

\noi We want to show that (i) and (ii') are equivalent. 

\noi(i) implies (ii'). We prove $h(S_x)\ne E$ by noting that $h(S_x)=E$ we would entail $h(x)=h(y)$ for some $y<x$, contradiction. To prove $h(x)=\min(h(S_x)^c)$, assume by contradiction that we have $h(x)\ne\min(h(S_x)^c)=:e$. If $h(x)<e$, then $h(x)\notin h(S_x)^c$, that is, $h(x)\in h(S_x)$, and we reach a contradiction as above. If $e<h(x)$, then $e=h(y)$ for some $y<x$, that is, $\min(h(S_x)^c)=e=h(y)\in h(S_x)$, contradiction. 

\noi(ii') implies (i). We assume (ii'), and, in particular, that $h$ is weakly increasing. To show that $h$ is increasing, suppose $x<y$ and $h(x)=h(y)$ (we cannot have $h(x)>h(y)$ because $h$ is weakly increasing). Since $h(x)=h(y)=\min(h(S_y)^c)$, we have $h(x)\in h(S_y)^c$, but $h(x)\in h(S_y)$, contradiction. Finally, $h(J)$ is downward closed because $e<h(x)=\min(h(S_x)^c)$ implies $e\in h(S_x)\subset h(J)$. 

In the statement of the Exercise, the condition that $J$ is well-ordered can be changed from an assumption to a conclusion. 

\noi(b) Let $a$ be in $E$, and assume there is an isomorphism of well-ordered sets $h:S_a\to E$. It suffices to derive a contradiction. Let $i:S_a\to E$ be the inclusion. By (a) $h$ and $i$ satisfy the same recursion relation. By the Theorem about the General Principle of Definition by Recursion, we have $h=i$, and thus $a\in h(S_a)=i(S_a)=S_a$, contradiction. 

\hs 

\noi$\bullet$ \textbf{Supplementary Exercise 11.3 p. 73 of the book.} Let $J$ and $E$ be well-ordered sets; suppose there is an order-preserving map $k: J \to E$. Using Exercises 1 and 2, show that $J$ has the order type of $E$ or a section of $E$. [Hint: Choose $e_0 \in E$. Define $h: J \to E$ by the recursion formula
\[h(\alpha) = \text{smallest }[E - h(S_\alpha)] \text{ if } h(S_\alpha) \neq E,\]
and $h(\alpha) = e_0$ otherwise. Show that $h(\alpha) \leq k(\alpha)$ for all $\alpha$; conclude that $h(S_\alpha) \neq E$ for all $\alpha$.] 

\noi\textbf{Solution.} We can assume $E\ne\varnothing$. Let $e_0$ be in $E$. Let $x$ be in $J$. We define $h:J\to E$ as in the hint. 

\noi Claim 1: $h(x)\le k(x)$ for all $x$. 

\noi Claim 2: $h(y)\le k(y)$ for all $y$ in $S_x$ implies $h(S_x)\ne E$. 

\noi Proof of Claim 2. For all $y$ in $S_x$ we have $h(y)\le k(y)<k(x)$, and in particular $k(x)\ne h(y)$. This implies $k(x)\notin h(S_x)$. 

\noi Proof of Claim 1. Assume by contradiction $h(x)>k(x)$ for some $x$. We can assume that $x$ is minimum for this condition. For $y<x$ we have $h(y)\le k(y)$, hence $h(S_x)\ne E$ by Claim~2.  

Claims 1 and 2 imply $h(x)=\min(h(S_x)^c)$ for all $x$, hence $h$ is increasing and $h(J)$ is downward closed by Supplementary Exercise 11.2 above, hence $J$ has the order type of $E$ or a section of $E$. 

\hs

\noi$\bullet$ \textbf{Supplementary Exercise 11.4 p. 73 of the book.} Use Exercises 1--3 to prove the following:
\begin{enumerate}
    \item[(a)] If $A$ and $B$ are well-ordered sets, then exactly one of the following three conditions holds: $A$ and $B$ have the same order type, or $A$ has the order type of a section of $B$, or $B$ has the order type of a section of $A$. [Hint: Form a well-ordered set containing both $A$ and $B$, as in Exercise 8 of §10; then apply the preceding exercise.]
    \item[(b)] Suppose that $A$ and $B$ are well-ordered sets that are uncountable, such that every section of $A$ and of $B$ is countable. Show $A$ and $B$ have the same order type.
\end{enumerate} 

\noi\textbf{Solution.} (a) For any element $x$ of any ordered set $X$, let $X_{<x}$ denote the corresponding section, and let us set $X_{<\infty}:=X$. Let $C$ be the well-ordered set containing both $A$ and $B$, described in Exercise~8 of §10, and let $k:A\to C$ and $\ell:B\to C$ be the natural increasing maps. By the previous Exercise, we have isomorphisms $A\simeq C_{<x}$ and $B\simeq C_{<y}$ for some $x$ and $y$ in $C\cup\{\infty\}$. We can assume $x\le y$. Then $x=y$ implies $A\simeq B$. If $x<y$, we get 
$$
A\simeq C_{<x}=(C_{<y})_{<x}\subsetneqq C_{<y}\simeq B.
$$ 
This implies $A\simeq B_{<b}$ for some $b$ in $B$. The fact that the various cases are exclusive follows from Supplementary Exercise 11.2b. 

\noi(b) Follows from (a). 

Here is an important consequence of (a): 

\begin{thm}[Comparability Theorem] 
If $A$ and $B$ are sets, then exactly one of the following three conditions holds: 

\emph{(i)} there is a bijection $A\to B$, 

\emph{(ii)} there is an injection $A\to B$ and a surjection $B\to A$, 

\emph{(iii)} there is an injection $B\to A$ and a surjection $A\to B$. 
\end{thm}

\hrule\medskip

\noi$\bullet$ \textbf{Supplementary Exercise 11.5 p. 73 of the book.} Let $X$ be a set; let $\mc A$ be the collection of all pairs $(A, <)$, where $A$ is a subset of $X$ and $<$ is a well-ordering of $A$. Define
\[(A, <) \prec (A',<')\]
if $(A, <)$ equals a section of $(A',<')$.
\begin{enumerate}
    \item[(a)] Show that $\prec$ is a strict partial order on $A$.
    \item[(b)] Let $\mc B$ be a subcollection of $A$ that is simply ordered by $\prec$. Define $B'$ to be the union of the sets $B$, for all $(B, <) \in\mc B$; and define $<'$ to be the union of the relations $<$, for all $(B, <) \in\mc B$. Show that $(B', <')$ is a well-ordered set.
\end{enumerate} 

\noi\textbf{Solution.} Left to the reader. 

\hs

\noi$\bullet$ \textbf{Supplementary Exercise 11.6 p. 73 of the book.} Use Exercises 1 and 5 to prove the following: 

\noi\textbf{Theorem.} The maximum principle is equivalent to the well-ordering theorem. 

\noi\textbf{Solution.} The fact that the well-ordering theorem implies the maximum principle is proved on p.~70 of the book. Let us prove the converse. In the setting of Supplementary Exercise~11.5b, take $\mc B$ maximal. Then it suffices to show that $B'=X$. If it was not so, we could add to $B'$ a new element $x$ and make it the largest element of $B'\cup\{x\}$, which would then be a well-ordered set larger than $B'$, contradiction. 

\hs 

\noi$\bullet$ \textbf{Supplementary Exercise 11.7 p. 73 of the book.} Use Exercises 1--5 to prove the following: 

\noi\textbf{Theorem.} The choice axiom is equivalent to the well-ordering theorem.

\noi Proof. Let $X$ be a set; let $c$ be a fixed choice function for the nonempty subsets of $X$. If $T$ is a subset of $X$ and $<$ is a relation on $T$, we say that $(T, <)$ is a tower in $X$ if $<$ is a well-ordering of $T$ and if for each $x \in T$,
\[x = c(X - S_x(T)),\]
where $S_x(T)$ is the section of $T$ by $x$.
\begin{enumerate}
    \item[(a)] Let $(T_1, <_1)$ and $(T_2, <_2)$ be two towers in $X$. Show that either these two ordered sets are the same, or one equals a section of the other. [Hint: Switching indices if necessary, we can assume that $h: T_1 \to T_2$ is order preserving and $h(T_1)$ equals either $T_2$ or a section of $T_2$. Use Exercise~2 to show that $h(x) = x$ for all $x$.]
    \item[(b)] If $(T, <)$ is a tower in $X$ and $T \neq X$, show there is a tower in $X$ of which $(T, <)$ is a section.
    \item[(c)] Let $\{(T_k, <_k)|k \in K\}$ be the collection of all towers in $X$. Let
    \[T = \bigcup_{k\in K} T_k \text{ and } <\ = \bigcup_{k\in K} (<_k).\]
    Show that $(T, <)$ is a tower in $X$. Conclude that $T = X$.
\end{enumerate} 

\noi\textbf{Solution.} (a) The map $h$ in the hint exists by Supplementary Exercise~4a. Let us show $h(x)=x$ for all $x$ thanks to Supplementary Exercise~2. Assume by contradiction that we have $h(x)\ne x$ for some $x$ in $T_1$, which we can suppose to be minimum for this condition. The map $h$ induces an isomorphism $T_1\simeq h(T_1)$, implying $h(T_{1,<y})=T_{2,h(y)}$ for all $y$ in $T_1$. By the choice of $x$ we get $T_{1,<x}=T_{2,<h(x)}$. Since $T_1$ and $T_2$ are towers, this entails 
$$
h(x)=c(X\sm T_{2,<h(x)})=c(X\sm T_{1,<x})=x,
$$ 
contradiction. The fact that $h(x)=x$ for all $x$ in $T_1$ implies, by Supplementary Exercise 11.2a, that $T_1$ is contained and downward closed in $T_2$. 

\noi(b) Add $c(X\sm T)$ to $T$, and make it the largest element. 

\noi(c) Left to the reader. 

This shows that the choice axiom implies the well-ordering theorem. The converse is clear. 

\section{Chapter 2. Topological Spaces and Continuous Functions}

\noi\textbf{Lemma 13.1 p. 80 of the book.} Recall the statement: 

\begin{lem}[Lemma 13.1 p. 80 of the book]\label{L13.1}
Let $X$ be a set; let $\mathcal{B}$ be a basis for a topology $\mathcal{T}$ on $X$. Then $\mathcal{T}$ equals the collection of all unions of elements of $\mathcal{B}$.
\end{lem}

Here is the proof given in the book: 

\begin{proof}
Given a collection of elements of $\mathcal{B}$, they are also elements of $\mathcal{T}$. Because $\mathcal{T}$ is a topology, their union is in $\mathcal{T}$. Conversely, given $U \in \mathcal{T}$, choose for each $x\in U$ an element $B_x$ of $\mathcal{B}$ such that $x\in B_x \subset U$. Then $U=\bigcup_{x\in U}B_x$, so $U$ equals a union of elements of $\mathcal{B}$.
\end{proof} 

To avoid the Axiom of Choice when showing that $U \in \mathcal{T}$ is a union of elements of $\mathcal{B}$, just note that $U$ is the union of all those $B$ in $\mc B$ which are contained in  $U$. 

%This kind of unnecessary use of the Axiom of Choice occurs several times in the book. To avoid repetitions we state the following Lemma. \begin{lem}[Avoiding the Axiom of Choice]\label{LAOC}Let $X$ be a topological space, $A$ a subset. Then $A$ is open if and only if for all $x$ in $A$ there is an open subset $U$ such that $x\in U\subset A$. \end{lem} 

\hs

\noi\textbf{About Lemma 13.2 p. 80 of the book.} Recall the statement: 

\begin{lem}[Lemma 13.2 of the book]%\label{L13.2a}
Let $X$ be a topological space. Suppose that $\mc C$ is a collection of open sets of X such that for each open set $U$ of $X$ and each $x$ in $U$, there is an element $C$ of $\mc C$ such that $x\in C\subset U$. Then $\mc C$ is a basis for the topology of $X$. 
\end{lem} 

The proof shows the following: 

\begin{lem}\label{L13.2b}
Let $X$ be a topological space and $\B$ a basis. Suppose that $\mc C$ is a set of open sets of X such that for each basic set $B$ in $\B$ and each $x$ in $B$, there is an element $C$ of $\mc C$ such that $x\in C\subset B$. Then $\mc C$ is also a basis for the topology of $X$. 
\end{lem} 

\smallskip\hrule\medskip 

\noi$\bullet$ \textbf{Solution to Exercise 13.6 p. 83 of the book.} We must show that the topologies $\mc T_\ell$ and $\mc T_K$ are incomparable. 

\noi Claim: $[2,3)\notin\mc T_K$. Proof. If not we would have $2\in(a,b)\sm K\subset[2,3)$ for some $a$ and $b$, hence $a<2$ and $a\le2$, contradiction. 

\noi Claim: $(-1,1)\sm K\notin\mc T_\ell$. Proof. If not we would have $0\in[a,b)\subset(-1,1)\sm K\subset[2,3)$ for some $a$ and $b$, hence $a\le0<b$, hence $a<\frac1n<b$ for some $n$, contradiction. 

\hs

\noi$\bullet$ \textbf{Solution to Exercise 13.7 p. 83 of the book.} Let us use the following notation: 

$\T_s:=$ standard topology, 

$\T_K:=$ topology of $\R_K$, 

$\T_{fc}:=$ finite complement topology, 

$\T_u:=$ upper limit topology (having the sets $(a,b]$ as basis), 

$\T_\infty:=$ topology having the sets $(-\infty,a)$ as basis. 

\noi We denote the corresponding topological spaces by $\R_s,\R_K,\R_{fc},\R_u$ and $\R_\infty$. Finally we write $\B_s,\B_K,\B_u$ and $\B_\infty$ for the obvious bases. 

The inclusions between these five topologies on $\R$ can be summarized by the diagram 
$$
\begin{matrix}
&u\\ 
&K\\ 
&s\\ 
fc&&\infty,
\end{matrix}
$$ 
where ``$i$ below $j$'' means ``$\T_i\subsetneqq\T_j$'', and ``$i$ and $j$ on the same level'' means ``$\T_i$ and $\T_j$ are incomparable''. 

Preliminary comments: It is easy to see that the elements of $\T_\infty$ are $\varnothing$, the intervals $(-\infty,a)$, and $\R$, and to observe that $\T_\infty\cap\T_{fc}=\{\varnothing,\R\}$. It is also easy to compare the standard topology $\T_s$ to the others: the elements of $\T_{fc}$ and $\T_\infty$ are clearly open in $\R_s$, and it is plain that the intervals $(a,b)$ (which are the elements on $\B_s$) are open in $\R_K$ and in $\R_\infty$ (note that $(a,b)=\bigcup_{d<b}(a,d])$. Clearly, $(-1,1)\sm K\in\T_K$ and $(a,b]\in\T_u$ are not open in $\R_s$. Moreover $(2,3]$ is in $\T_u$ but not in $\T_K$. So, it only remains to prove $\T_K\subset\T_u$. 

Let $x$ be in $(a,b)\sm K$. It suffices to show that there is a $c$ such that $x\in(c,x]\subset(a,b)\sm K$. If $x\le0$ we set $c:=a$. If $\frac1{n+1}<x<\frac1n$ we set $c:=\frac1{n+1}\,$. If $x>1$ we set $c:=\max(1,a)$. 

\hs

\noi\textbf{Exercise 13.8c p. 83 of the book.} Show that the collection 
$$
\mc C=\{[a,b)\ |\ a<b,\ a \text{ and $b$ rational}\}
$$ 
is a basis that generates a topology different from the lower limit topology on $\R$. 

\noi\textbf{Solution.} It is easy to check that $\mc C$ is indeed a basis. Let us show that it generates a topology different from the lower limit topology. Otherwise we would have $[\sqrt2,2)=\bigcup_{i\in I}[a_i,b_i)$ with $a_i<b_i$ and $a_i$ and $b_i$ rational for all $i$. This implies $a_i\ge\sqrt2$, that is $a_i>\sqrt2$, for all $i$, hence $\sqrt2\notin\bigcup_{i\in I}[a_i,b_i)$, contradiction. 

\hs

\noi\textbf{Subspace topology p. 88 of the book.} Munkres writes: 

\noi\textbf{Definition.} Let $X$ be a topological space with topology $\mathcal{T}$. If $Y$ is a subset of $X$, the collection
\[
\mathcal{T}_Y = \{Y \cap U \mid U \in \mathcal{T}\}
\]
is a topology on $Y$, called the \textbf{subspace topology}. With this topology, $Y$ is called a \textbf{subspace} of $X$; its open sets consist of all intersections of open sets of $X$ with $Y$. %\vspace{10pt}

It is easy to see that $\mathcal{T}_Y$ is a topology. It contains $\varnothing$ and $Y$ because
\[
\varnothing = Y \cap \varnothing \quad \text{and} \quad Y = Y \cap X,
\]
where $\varnothing$ and $X$ are elements of $\mathcal{T}$. The fact that it is closed under finite intersections and arbitrary unions follows from the equations
\[
(U_1 \cap Y) \cap \cdots \cap (U_n \cap Y) = (U_1 \cap \cdots \cap U_n) \cap Y,
\]
\[
\bigcup_{\alpha \in J} (U_\alpha \cap Y) = \left(\bigcup_{\alpha \in J} U_\alpha\right) \cap Y.
\] 

\begin{proof}
Given a collection of elements of $\mathcal{B}$, they are also elements of $\mathcal{T}$. Because $\mathcal{T}$ is a topology, their union is in $\mathcal{T}$. Conversely, given $U \in \mathcal{T}$, choose for each $x \in U$ an element $B_x$ of $\mathcal{B}$ such that $x \in B_x \subset U$. Then $U = \bigcup_{x\in U} B_x$, so $U$ equals a union of elements of $\mathcal{B}$.
\end{proof} 

\noi End of the excerpt. 

The Axiom of Choice is used to handle arbitrary unions. One can avoid it by proceeding as follows. 

Given $A\subset Y\subset X$, set $\mc U:=\{U\in\mc T\ |\ U\cap Y\subset A\}$, and define the open subset $U_A$ of $X$ by $U_A:=\bigcup_{U\in\mc U}U$. Then we have 
$$
A\subset U_A\iff U_A\cap Y=A.
$$ 
Proof: We have $Y\cap U_A=Y\cap\bigcup_{U\in\mc U}U=\bigcup_{U\in\mc U}Y\cap U\subset A$, so 
$$A=U_A\cap Y\iff A\subset U_A\cap Y\iff A\subset U_A.\qquad\square
$$
And we decree that $A$ is open in $Y$ if and only if the above equivalent conditions are satisfied. This is equivalent to the usual definition. Proof: If $A=U_A\cap Y$, then $A$ is open in $Y$ in the usual sense. If $A=U\cap Y$ for some $U$ open in $X$, then $U\in\mc U$, hence $U\subset U_A$, and we get $A=U\cap Y\subset U_A\cap Y\subset A$, hence $A=U_A\cap Y$. Observe that, in general, $U_A\cap Y$ is the \textbf{interior} of $A$ in $Y$. 

Note that things are even nicer if we use closed subsets instead of open ones. Indeed, given $A\subset Y\subset X$ as above, there is a least closed subset $C_A$ of $X$ such that $A\subset C_A\cap Y$, and $C_A$ is the closure of $A$ in $X$. What I find remarkable is that $C_A$ depends only on $A$ and $X$, but not on $Y$. (I don't know if there is a conceptual reason for that.) (To see that $U_A$ depends on $Y$ in general, let $X$ be nonempty and let $A$ be empty. If $Y=\varnothing$, then $U_A=X$, but if $Y=X$, then $U_A=\varnothing$.) \bigskip 

\hrule\medskip

\noi\textbf{Exercise 16.5 p. 92 of the book.} Let $X$ and $X'$ denote a single set in the topologies $\mc T$ and $\mc T'$, respectively; let $Y$ and $Y'$ denote a single set in the topologies $\mc U$ and $\mc U'$, respectively. Assume these sets are nonempty.

\noi(a) Show that if $\mc T'\supset\mc T$ and $\mc U'\supset\mc U$, then the product topology on $X'\times Y'$ is finer than the product topology on $X\times Y$.

\noi(b) Does the converse of (a) hold? Justify your answer.

\noi\textbf{Solution.} Part (a) is straightforward. The answer to the question in (b) is Yes. Here is the justification. We denote the respective product topologies by $\mc V$ and $\mc V'$.  Assume $\mc V'\supset\mc V$. It suffices to show $\mc T'\supset\mc T$. Let $U$ be in $\mc T$. It is enough to prove $U\in\mc T'$. The set $U\times Y$ is in $\mc V\subset\mc V'$, that is, 
$$
U\times Y=\bigcup_{i\in I}\ (U'_i\times V'_i)
$$ 
with $U'_i\in\mc T'$ and $V'_i\in\mc U'$ for all $i$. It suffices to show $U=\bigcup_{i\in I}U'_i$. Let $u$ be in $U$. Pick some $y$ in $Y$. Then $u\times y=u'_i\times v'_i$ for some $i$ and some $u'_i\in U'_i$ and $v'_i\in V'_i$, so $u=u'_i$ is in $U'_i$. Conversely, let $u'_i$ be in $U'_i$ for some $i$. Then $u'_i\times v'_i=u\times y$ for some $u$ in $U$ and $y$ in $Y$. In particular $u'_i=u\in U$. 

\hs

\noi\textbf{Exercise 16.7 p. 92 of the book.} Let $X$ be an ordered set. If $Y$ is a proper subset of $X$ that is convex in $X$, does it follow that $Y$ is an interval or a ray in $X$? 

\noi\textbf{Solution.} No. Example: $X:=\{1,2\}\times\Z$ with the dictionary order, $Y:=\Z\times\{1\}$. 

\hs

\noi\textbf{Exercise 16.8 p. 92 of the book.} If $L$ is a straight line in the plane, describe the topology $L$ inherits as a subspace of $\R_\ell\times\R$ and as a subspace of $\R_\ell\times\R_\ell$. In each case it is a familiar topology. 

\noi\textbf{Solution.} Let $\T_1$ be the topology of $\R_\ell\times\R$, let $\T_2$ be the topology of $\R_\ell\times\R_\ell$, and let $L_i$, for $i=1,2$, be the line $L$ equipped with the topology induces by $\T_i$. I think what Munkres wants us to realize is that $L_i$ is \textbf{homeomorphic} to $\R$ with the standard topology, or to $\R$ with the lower limit topology, or to $\R$ with the discrete topology, depending on $i$ and the direction of $L$. But of course the word ``homeomorphic'' is introduced much later in the book, so, strictly speaking, the question, as stated, does not make sense. We will solve the following interpretation of the exercise: 

\noi\textbf{Exercise 16.8'.} For $i=1,2$ let $p_i:\R^2\to\R$ be the $i$th canonical projection. We set $i(L):=2$ if $L$ is vertical, and $i(L)=1$ otherwise. Then the restriction $r_L:L\to\R$ of $p_{i(L)}$ to $L$ is bijective. Let $\T_o$ be the standard topology of $\R$ (the subscript ``o'' stands for ``order topology''), let $\R_o$ be $\R$ equipped with $\T_o$, let $\T_\ell$ be the lower limit topology of $\R$, let $\R_\ell$ be $\R$ equipped with $\T_\ell$, let $\T_{\ell o}$ be the topology of $\R_\ell\times\R_o$, let $\T_{\ell\ell}$ be the topology of $\R_\ell\times\R_\ell$, let $\mc U_{\ell o}$ be the topology induced on $L$ by $\T_{\ell o}$, let $\mc U_{\ell\ell}$ be the topology induced on $L$ by $\T_{\ell\ell}$, let $\T_{\ell oL}$ be the topology on $\R$ obtained by transporting $\T_{\ell o}$ along $r_L$, and let $\T_{\ell\ell L}$ be the topology on $\R$ obtained by transporting $\T_{\ell\ell}$ along $r_L$. Describe the topologies $\T_{\ell oL}$ and $\T_{\ell\ell L}$. 

\noi\textbf{Solution to Exercise 16.8'.} We decree that the slope of a vertical line is $+\infty$. Let $s\in\R\cup\{+\infty\}$ be the slope of $L$, and let $\T_d$ be the discrete topology of $\R$. We claim 
$$
\T_{\ell oL}=
\begin{cases}
\T_o&\text{if }s=+\infty\\ 
\T_\ell&\text{if }s\in\R,
\end{cases}
$$ 
$$
\T_{\ell\ell L}=
\begin{cases}
\T_d&\text{if }s<0\\ 
\T_\ell&\text{otherwise.}
\end{cases}
$$ 
To prove this, we first analyze the intersections 
$$
L\cap\big(([a,b)\times(c,d)\big)\text{ and }L\cap\big(([a,b)\times[c,d)\big)
$$ 
for $a,b,c,d\in\R,a<b,c<d$, where $L$ is fixed and $a,b,c$ and $d$ vary. We obtain a set of subsets of $L$, which we transport to $\R$ by $r_L$, and we take the topology on $\R$ generated by these subsets. The details are tedious, but very easy, and left to the reader. 

\hs

\noi\textbf{Exercise 16.9 p. 92 of the book.} Show that the dictionary order topology on the set $\R\times\R$ is the same as the product topology $\R_d\times\R$, where $\R_d$ denotes $\R$ in the discrete topology. Compare this topology with the standard topology on $\R^2$. 

\noi\textbf{Solution.} First recall the definition of the order topology: 

Let $X$ be a set with a simple order relation; assume $X$ has more than one element. Let $\mathcal{B}$ be the collection of all sets of the following types:
\begin{enumerate}
    \item All open intervals $(a,b)$ in $X$.
    \item All intervals of the form $[a_0,b)$, where $a_0$ is the smallest element (if any) of $X$.
    \item All intervals of the form $(a,b_0]$, where $b_0$ is the largest element (if any) of $X$.
\end{enumerate}
The collection $\mathcal{B}$ is a basis for a topology on $X$, which is called the \textbf{order topology}. 

The dictionary order topology on $\R\times\R$ is given by the basis elements $(a\times b,c\times d)$ with $a\times b<c\times d$. One checks easily that the intervals of the form $(a\times b,a\times c)$ with $b<c$ also form a basis for the order topology (see Example~2 p.~85 of the book). This shows that the dictionary order topology on the set $\R\times\R$ is the same as the product topology $\R_d\times\R$. It is strictly finer than the standard topology on $\R^2$. 

\hs

% 2411180941 https://docs.google.com/document/d/10xdrlyZeC-5pva5J1qbJltRwUEIVd6Eaotx7jakS8ZA/edit?tab=t.0

\noi\textbf{Exercise 16.10 p. 92 of the book.} Let $I=[0,1]$. Compare the product topology on $I\times I$, the dictionary order topology on $I\times I$, and the topology $I\times I$ inherits as a subspace of $\R\times\R$ in the dictionary order topology. 

\noi\textbf{Solution.} Several topological spaces $X$ are (explicitly or implicitly) involved in the above statement. We denote the topology of such an $X$ by $\T(X)$, and if $\T(X)$ comes with a preferred basis, we denote it by $\B(X)$. If $X$ is an ordered set, we denote by $X_o$ the set $X$ equipped with the order topology $\T(X_o)$, and by $\B(X_o)$ the basis defining $\mc T(X_o)$ (see the Solution to Exercise~16.9 p.~92 above). If $S$ is a set, we denote by $S_d$ the set $S$ equipped with the discrete topology, and we regard the set $\mc B(S_d)$ of all singletons contained in $S$ as the preferred basis of $\mc T(S_d)$. If $X$ and $Y$ are topological spaces with preferred basis $\B(X)$ and $\B(Y)$, we define $\B(X\times Y)$ as the basis consisting of the $B\times C$ with $B\in\mc B(X)$ and $C\in\mc B(Y)$ (see Theorem~15.1 p.~86). If $X$ is a topological space with preferred basis $\B(X)$ and $Y$ is a subspace of $X$, we define $\B(Y)$ as the basis consisting of the $B\cap Y$ with $B\in\mc B(X)$ (see Theorem~16.1 p.~89). If $X$ and $Y$ are ordered set, we denote by $(X\times Y)_o$ the set $X\times Y$ equipped with the dictionary order topology. 

Recall the following facts from the book. By Example~1 p.~85, $\R$ equipped with its standard topology coincides with $\R_o$. By Example~1 p.~90, $I\subset\R$ equipped with the subspace topology is equal to $I_o$. In Example~3 p.~90, $(I^2)_o$ is denoted $I_o^2$, and is called the \textbf{ordered square}. Remember also that in the Solution to Exercise~16.9 p.~92 we saw that $(\R^2)_o=\R_d\times\R_o$ (equipped with the product topology). 

Using these facts and Theorem~16.3 p.~89, it is easy to see that the three topological spaces of the exercise are respectively $(I_o)^2,(I^2)_o$ and $I_d\times I_o$. In particular, the six sets 
$$
\T((I_o)^2),\quad\T((I^2)_o),\quad\T(I_d\times I_o),\quad\B((I_o)^2),\quad\B((I^2)_o),\quad\B(I_d\times I_o)
$$ 
are well-defined. 

We claim 
$$
\mc T((I_o)^2)\cup\mc T((I^2)_o)\subsetneqq\mc T(I_d\times I_o)\quad\text{and}\quad\mc T((I_o)^2)\not\subset\mc T((I^2)_o)\not\subset\mc T((I_o)^2).
$$ 
Let $U\subset I\times I$. Then $U\in\mc T(I_d\times I_o)$ if and only if 
$$
U=\bigcup_{x\in I}\ (\{x\}\times U_x)
$$ 
for some family $(U_x)_{x\in I}$ of members of $\mc T(I_o)$. 

The inclusion $\mc T((I_o)^2)\cup\mc T((I^2)_o)\subset\mc T(I_d\times I_o)$ follows immediately from the above observations. 

The following criterium to prove that a subset of $I^2$ is not in $\mc T((I^2)_o$ will be handy. 

\noi Criterium: If $U$ is an open subset of $(I^2)_o$ containing $0\times1$, then $U$ contains $\varepsilon\times0$ for some $\varepsilon\in I$. 

Thanks to this criterium, we see that 
$$
\{0\}\times I=[0\times0,0\times1]\in\mc T(I_d\times I_o)\sm(\mc T((I_o)^2)\cup\mc T((I^2)_o)),
$$ 
implying $\mc T((I_o)^2)\cup\mc T((I^2)_o)\ne\mc T(I_d\times I_o)$. 

A similar argument shows that $I\times(0,1]\in\mc T((I_o)^2)\sm\mc T((I^2)_o)$, and thus that $\mc T((I_o)^2)\not\subset\mc T((I^2)_o)$.  

Finally, to prove $\mc T((I^2)_o)\not\subset\mc T((I_o)^2)$, note that $\{0\}\times(0,1)=(0\times0,0\times1)\in\mc T((I^2)_o)\sm\mc T((I_o)^2)$. 

\hs

% removed https://docs.google.com/document/d/18jcH4K8A9CRBgWSgQIAbqhm3ySI4aYJfhapsOAo0axs/edit?tab=t.0

\noi\textbf{Theorem 17.2 p. 94 of the book.} We give a slightly different proof of the indicated Theorem. First recall the statement. 

\begin{thm}[Theorem 17.2 p. 94 of the book]\label{T17.2} 
Let $X$ be a topological space, $Y$ a subspace, and $A$ a subset of $Y$. Then $A$ is closed in $Y$ if and only $A=Y\cap C$ for some closed subset of $X$. 
\end{thm} 

\begin{proof} 
It suffices to show that $Y\sm A=Y\cap U$, where $U$ is an open subset of $X$, if and only if $A=Y\cap(X\sm U)$. Hence it is enough to prove that we have $Y\sm(Y\cap U)=Y\cap(X\sm U)$, or equivalently $Y\sm U=Y\cap(X\sm U)$, for all subset $U$ of $X$. But this is clear. 
\end{proof}

\hrule\medskip

\noi\textbf{Theorem 17.4 p. 95 of the book.} Recall the statement: 

\begin{thm}[Theorem 17.4 p. 95 of the book] 
Let $X$ be a topological space, $Y$ a subspace, and $A$ a subset of $Y$. Then the closure $\overline A^Y$ of $A$ in $Y$ equals $\overline A\cap Y$, where $\overline A$ is the closure of $A$ in $X$.
\end{thm} 

Here is a slightly different proof. 

\begin{proof} 
Set 
$$
\mc B=\{B\ |\ B\text{ closed in }Y,\ B\supset A\}.
$$ 
For all $B$ in $\mc B$ put 
$$
\mc C(B)=\{C\ |\ C\text{ closed in }X,\ C\supset B\}.
$$ 
Finally write 
$$
\mc C=\{C\ |\ C\text{ closed in }X,\ C\supset A\}.
$$ 
We get 
$$
\mc C=\bigcup_{B\in\mc B}\mc C(B)
$$ 
and 
$$
\overline A^Y=\bigcap_{B\in\mc B}B=\bigcap_{B\in\mc B}\bigcap_{C\in\mc C(B)}(Y\cap C)=\bigcap_{C\in\mc C}(Y\cap C)=Y\cap\bigcap_{C\in\mc C}C=Y\cap\overline A.
$$ 
\end{proof}

\hrule\medskip

\noi\textbf{Theorem 17.6 p. 97 of the book.} Recall the statement: % variant https://docs.google.com/document/d/17l98t_t6zXfymBp91exwc1XqfGgzNXKYvZzOtOorNFg/edit?tab=t.0

\begin{thm}[Theorem 17.6 p. 97 of the book]%\label{T17.6} 
Let $A$ be a subset of the topological space $X$; let $A'$ be the set of all limit points of $A$. Then $\overline A=A\cup A'$. 
\end{thm} 

Here is a slightly different proof. 

\begin{proof}
In this proof we denote the closure $\overline B$ of a subset $B$ of $X$ by $B^-$. To show $A'\subset A^-$, note that if $x$ is in $A'$, then $x$ is in $(A\sm\{x\})^-\subset A^-$. It only remains to prove $A^-\subset A\cup A'$. Assume by contradiction that there is an $x$ in $A^-$ which is not in $A\cup A'$. We have in particular $x\notin(A\sm\{x\})^-$, that is, $x\in U:=X\sm(A\sm\{x\})^-$ with $U$ open. The fact that $x$ is in $\overline A$ implies that $U$ intersects $A$. Let $a$ be such an intersection point. Since $x$ is not in $A$ but $a$ is, we have $a\ne x$, hence $a$ is in $A\sm\{x\}$, which is disjoint from $U$, contradiction. 
\end{proof} 

\hrule\medskip

\noi\textbf{Theorem 17.8 p. 99 of the book.} Recall the statement: 

\begin{thm}[Theorem 17.8 p. 99 of the book]%\label{T17.8} 
Every finite point set in a Hausdorff space $X$ is closed.
\end{thm} 

Here is a slightly different proof. 

\begin{proof}
Let $x_0$ be a point of $X$, let $\mc U$ be the set of all those open subsets of $X$ which do not contain $x_0$, let $U$ be the union of the members of $\mc U$, and let $x$ be in $X\sm\{x_0\}$. It suffices to show that $x\in U$, that is, $x\in V$ for some $V$ in $\mc U$, which is clear. 
\end{proof} 

\hrule\medskip

\noi\textbf{Theorem 17.9 p. 99 of the book.} Recall the statement: 

\begin{thm}[Theorem 17.8 p. 99 of the book]%\label{T17.9} 
Let $X$ be a space satisfying the $T_1$ axiom; let $A$ be a subset of $X$. Then the point $x$ is a limit point of $A$ if and only if every neighborhood of $x$ contains infinitely many points of $A$.
\end{thm} 

Here is a slightly different proof. 

\begin{proof}
It suffices to show that the following two conditions are equivalent: 

(1) there is a neighborhood $U$ of $x$ such that $U\cap A=\{x\}$, 

(2) there is a neighborhood $V$ of $x$ such that $V\cap A$ is finite. 

\noi Clearly (1) implies (2). To prove the converse it suffices to set $U:=(V\sm A)\cup\{x\}$. (Since $X$ is $T_1$, the subset $V$, being obtained from $U$ by removing finitely many points, is open.) 
\end{proof} 

\hrule\medskip

\noi\textbf{Exercise 17.3 p. 100 of the book.} Show that if $A$ is closed in $X$ and $B$ is closed in $Y$, then $A\times B$ is closed in $X\times Y$.

\noi\textbf{Solution.} Setting $U:=X\sm A,V:=Y\sm B$, we get
$$
(X\times Y)\sm(A\times B)=(X\times V)\cup(U\times Y).
$$ 
\hrule\medskip

\noi\textbf{Exercise 17.5 p. 100 of the book.} Let $X$ be an ordered set in the order topology. Show that $\overline{(a,b)}\subset[a,b]$. Under what conditions does equality hold?

\noi\textbf{Solution.} To prove the indicated inclusion, it suffices to show that 
\begin{equation}\label{E17.5}
[a,b]\text{ is closed.}
\end{equation} 
To prove \eqref{E17.5}, let $c$ be in $X\sm[a,b]$. Assume first $c<a$. If there is a $d$ less than $c$, then $(d,a)$ contains $c$, and is open and disjoint from $[a,b]$. If $c$ is the least element of $X$, then $[c,a)$ contains $c$, and is open and disjoint from $[a,b]$. The case $c>b$ is similar. 

For the second question, note that, by \eqref{E17.5}, the set $\overline{(a,b)}$ is equal to $(a,b)$, to $[a,b)$, to $(a,b]$, or to $[a,b]$. Thus it suffices to determine when $a$ or $b$ is in $\overline{(a,b)}$. We claim: 

(a) the point $a$ is not in $\overline{(a,b)}$ if and only if it has an immediate successor, 

(b) the point $b$ is not in $\overline{(a,b)}$ if and only if it has an immediate predecessor. 

\noi To prove this, we can, by Theorems 16.4 p.~91 and 17.4 p.~95, assume that $X=[a,b]$. To prove (a), note that 
$$
a\notin\overline{(a,b)}
$$ 
$$
\iff\text{there is a $c$ in $[a,b]$ with $a\in[a,c)$ and }[a,c)\cap(a,b)=\varnothing
$$ 
$$
\iff\text{there is a $c$ in $[a,b]$ which is the immediate successor of $a$.}
$$ 
The proof of (b) is similar. 

\hs

\noi\textbf{Exercise 17.6 p. 101 of the book.} Let $A, B$, and $A_{\alpha}$ denote subsets of a space $X$. Prove the following:
\begin{enumerate}
    \item[(a)] If $A \subset B$, then $\overline{A} \subset \overline{B}$.
    \item[(b)] $\overline{A \cup B}=\overline{A}\cup\overline{B}$.
    \item[(c)] $\overline{\bigcup A_{\alpha}} \supset \bigcup \overline{A}_{\alpha}$; give an example where equality fails.
\end{enumerate}

\noi\textbf{Solution.} (a) We have $A\subset\overline B$, and thus $\overline{A} \subset \overline{B}$. 

\noi(b) For $x$ in $X$ the following conditions are equivalent: 

\noi(A) $x\notin\overline{A\cup B}$, 

\noi(B) some neighborhood $U$ of $x$ does not intersect $A\cup B$, 

\noi(C) some neighborhood $U$ of $x$ intersects neither $A$ nor $B$, 

\noi(D) (some neighborhood $V$ of $x$ does not intersect $A$) and (some neighborhood $W$ of $x$ does not intersect $B$), 

\noi(E) $x\notin\overline{A}$ and $x\notin\overline{B}$, 

\noi(F) $x\notin\overline{A}\cup\overline{B}$,

\noi the implication (D)$\implies$(C) being obtained by setting $U:=V\cap W$. % previous version https://docs.google.com/document/d/1dG3nKbyB47SJOAJsVicdS2wlnvA0YCdtvl8Yt7BpTmw/edit?tab=t.0

\noi(c) For $x$ in $X$ we have: 
$$
x\notin\overline{\bigcup A_\alpha}
$$ 
$$
\implies\text{some neighborhood $U$ of $x$ does not intersect }\bigcup A_\alpha
$$ 
$$
\implies\text{some neighborhood $U$ of $x$ intersects no }A_\alpha
$$ 
$$
\implies x\notin\overline{A_\alpha}\text{ for all }\alpha
$$ 
$$
\implies x\notin\bigcup\overline{A_\alpha}.
$$ 

\hs 

\noi\textbf{Exercise 17.8 p. 108 of the book.} Let $A, B, \text{ and } A_\alpha$ denote subsets of a space $X$. Determine whether the following equations hold; if an equality fails, determine whether one of the inclusions $\subset$ or $\supset$ holds.
\begin{enumerate}%[(a)]
\item[(a)] $\overline{A \cap B}= \overline{A} \cap \overline{B}$
\item[(b)] $\overline{\bigcap A_\alpha}=\bigcap\overline{A_\alpha}$
\item[(c)] $\overline{A\sm B}=\overline{A}\sm\overline{B}$.
\end{enumerate}

\noi\textbf{Solution.} We claim that (b) holds, and will clearly imply (a). Proof: Let $x$ be in $X$. Then 
$$
x\in\overline{\bigcap A_\alpha}
$$ 
$$
\iff\text{ every neighborhood of $x$ intersects }\bigcap A_\alpha
$$ 
$$
\iff\text{ every neighborhood of $x$ intersects $A_\alpha$ for all $\alpha$}
$$ 
$$
\iff x\text{ is in }\overline{A_\alpha}\text{ for all }\alpha,
$$ 
$$
\iff x\text{ is in }\bigcap\overline{A_\alpha}.
$$ 
(c) If $X=A=\R$ and $B=\Q$, then we have $\overline{A\sm B}=\R$ and $\overline{A}\sm\overline{B}=\varnothing$. Thus our only hope is to have $\overline{A}\sm\overline{B}\subset\overline{A\sm B}$ for all $A,B$. We can rewrite this inclusion as $\overline{A}\cap\overline{B}^c\subset\overline{A\cap B^c}$ for all $A,B$, where $C^c$ means $X\sm C$. In view of (a), this is equivalent to $\overline{B}^c\subset\overline{B^c}$ for all $B$. To prove this, it suffices to show that $\overline{B}^c$ is the interior of $B^c$. Let $U$ be an open subset of $X$ contained in $B^c$. It is enough to prove $U\subset\overline{B}^c$. We have $U^c\supset B$. Since $U^c$ is closed, this implies $U^c\supset\overline B$, and thus $U\subset\overline{B}^c$. 

\hs

\noi\textbf{Exercise 17.9 p. 101 of the book.} Let $A \subset X$ and $B \subset Y$. Show that in the space $X\times Y$,
\[
\overline{A\times B}=\overline{A}\times\overline{B}.
\]

\noi\textbf{Solution.} We have $\overline{A\times B}\subset\overline{A}\times\overline{B}$ by Exercise 17.3. To prove the converse inclusion, let $z=x\times y$ be in $\overline{A}\times\overline{B}$, that is, $x\in\overline{A},y\in\overline{B}$, and let $W$ be an open subset of $X\times Y$ containing $z$. It suffices to show $W\cap(A\times B)\ne\varnothing$. There is an open subset $U$ of $X$ containing $x$ and an open subset $V$ of $Y$ containing $y$ such that $U\times V\subset W$. It is enough to prove $(U\times V)\cap(A\times B)\ne\varnothing$, that is $(U\cap A)\times(V\cap B)\ne\varnothing$. But this follows from the assumption that $x\in\overline{A}$ and $y\in\overline{B}$. 

\hs

\noi\textbf{Exercise 17.10 p. 101 of the book.} Show that every order topology is Hausdorff.

\noi\textbf{Solution.} Let $x,y$ be in $X$ with $x<y$. If there is a $z$ in $(x,y)$ we separates $x$ and $y$ with the open rays $(-\infty,z)$ and $(z,+\infty)$. If $(x,y)$ is empty, we separates $x$ and $y$ with the open rays $(-\infty,y)$ and $(z,+\infty)$. 

\hs 

%\noi\textbf{Exercise 17.11 p. 101 of the book.} Show that the product of two Hausdorff spaces is Hausdorff. 
%\noi\textbf{Exercise 17.12 p. 101 of the book.} Show that a subspace of a Hausdorff space is Hausdorff. 

\noi\textbf{Exercise 17.13 p. 101 of the book.} Show that $X$ is Hausdorff if and only if the \emph{diagonal} $\Delta=\{x\times x\mid x\in X\}$ is closed in $X\times X$. 

\noi\textbf{Solution.} Let $\overline\Delta$ be the closure of the diagonal, and let $z=x\times y$ be in $X^2$. We claim: $z\notin\overline\Delta\iff$ $x$ and $y$ can be separated by disjoint open sets. 

\noi$\implies$: There is an open subset $W$ of $X^2$ such that $z\in W$ and $W\cap\Delta=\varnothing$. There are open subsets $U$ and $V$ of $X$ such that $z\in U\times V\subset W$. We have $x\in U,y\in V$. If $u$ was in $U\cap V$, then $u\times u$ would be in $\Delta\cap W$, contradiction. 

\noi$\impliedby$: If there are disjoint open subsets $U$ and $V$ of $X$ such that $x\in U,y\in V$, then $z$ is in $U\times V$, and it suffices to show $(U\times V)\cap\Delta=\varnothing$. But $(U\times V)\cap\Delta$ is the diagonal of $U\cap V$, which is empty. 

\hs 

\noi\textbf{Exercise 17.14 p. 101 of the book.} In the finite complement topology on $\mathbb{R}$, to what point or points does the sequence $x_{n}=\frac1n$ converge? 

\noi\textbf{Solution.} Set $S:=\{\frac1n\ |\ n\in\Z_+\}$, and not that any nonempty open subset $U$ of $\R$ (with the finite complement topology) intersects $S$. This means that $\R$ is the set of limits of the sequence $\frac1n$. 

\hs 

\noi\textbf{Exercise 17.15 p. 101 of the book.} Show the $T_{1}$ axiom is equivalent to the condition that for each pair of points of $X$, each has a neighborhood not containing the other. 

\noi\textbf{Solution.} (In the statement of the Exercise, ``each has a neighborhood not containing the other'', means ``each point has a neighborhood not containing the other point''.) Recall that $X$ is $T_1$ if and only if the finite subsets of $X$ are closed. If $X$ is $T_1$ and $x$ and $y$ are distinct points of $X$, the open subsets $X\sm\{x\}$ and $X\sm\{y\}$ do the job. If $X$ is \emph{not} $T_1$, there distinct points $x$ and $y$ of $X$ such that $y$ is in the closure of $\{x\}$, and any neighborhood of $y$ contains $x$. 

\hs

\noi\textbf{Exercise 17.16 p. 101 of the book.} Consider the five topologies on $\mathbb{R}$ given in Exercise 7 of $\S 13$. 
    \begin{enumerate}
        \item[(a)] Determine the closure of the set $K=\left\{\frac1n\mid n\in \mathbb{Z}_{+}\right\}$ under each of these topologies.
        \item[(b)] Which of these topologies satisfy the Hausdorff axiom? the $T_{1}$ axiom?
    \end{enumerate}

\noi\textbf{Solution.} Recall that the five topologies are: 

$\T_s:=$ standard topology, 

$\T_K:=$ topology of $\R_K$, 

$\T_{fc}:=$ finite complement topology, 

$\T_u:=$ upper limit topology (having the sets $(a,b]$ as basis), 

$\T_\infty:=$ topology having the sets $(-\infty,a)$ as basis. 

\noi Recall also that $\T_K$ was defined as follows: Let $K$ be defined as above. The topology $\T_K$ generated by all open intervals $(a,b)$, along with all sets of the form $(a,b)\sm K$ will be called the $K$-topology on $\R$. When $\R$ is given this topology, we denote it by $\R_K$. We define $\R_s,\R_{fc},\R_u$ and $\R_\infty$ similarly. 

\noi(a) Let $C_*$ be the closure of $K$ in $\R_*$. Clearly, $C_s=C_\infty=K\cup\{0\}$ and $C_{fc}=\R$ (see Exercise~17.14). We claim $C_K=K$. Indeed, the inclusion $C_K\subset K\cup\{0\}$ is easy, and we have $0\notin C_u$ because 0 is in the open set $(-1,1)\sm K$, which is disjoint from $K$. We claim $C_u=K$. The argument is the same, with $(-1,0]$ instead of $(-1,1)\sm K$.  

\noi(b) Clearly $\R_s,\R_K$ and $\R_u$ are Hausdorff, and $\R_{fc}$ is $T_1$ but not Hausdorff. Finally, $\R_\infty$ is also $T_1$ but not Hausdorff. The fact that it is $T_1$ is clear. It is not Hausdorff because all the nonempty open sets are basic, and the intersection of any two basic sets is basic. 

\hs 

\noi\textbf{Exercise 17.17 p. 101 of the book.} Consider the lower limit topology on $\mathbb{R}$ and the topology given by the basis $\mathcal{C}$ of Exercise 8 of $\S 13$. Determine the closures of the intervals $A=(0,\sqrt{2})$ and $B=(\sqrt{2},3)$ in these two topologies. 

\noi\textbf{Solution.} Recall the definition of $\mc C$: 
$$
\mc C=\{[a,b)\ |\ a<b,\ a \text{ and $b$ rational}\}.
$$ 
For $\subset\R$ let ${\overline S\,}^\ell$ and ${\overline S\,}^{\mc C}$ denote respectively the closure in each of the two topologies in the statement. Then we have 
$$
{\overline{(0,\sqrt2)}\,}^\ell=[0,\sqrt2),\quad{\overline{(0,\sqrt2)}\,}^{\mc C}=[0,\sqrt2],\quad
{\overline{(\sqrt2,3)}\,}^\ell=[\sqrt2,3)={\overline{(\sqrt2,3)}\,}^{\mc C}
$$ 
The justifications are left to the reader. 

\hs

\noi\textbf{Exercise 17.18 p. 101 of the book.} Determine the closures of the following subsets of the ordered square:
    \[
    \begin{aligned}
    & A = \left\{\textstyle{\frac{1}{n}}\times 0 \mid n \in \mathbb{Z}_{+}\right\} \\
    & B = \left\{(1-\textstyle{\frac{1}{n}})\times\textstyle{\frac{1}{2}} \mid n \in \mathbb{Z}_{+}\right\} \\
    & C = \{x \times 0 \mid 0 < x < 1\} \\
    & D = \left\{x \times \textstyle{\frac{1}{2}} \mid 0 < x < 1\right\} \\
    & E = \left\{\textstyle{\frac{1}{2}} \times y \mid 0 < y < 1\right\}.
    \end{aligned}
    \]
\noi\textbf{Solution.} Let $X$ be the ordered square. For any $S\subset X$ we set $S':=\overline S\sm S$. We have 
$$
A'=\{0\times1\},\quad B'=\{1\times0\},\quad C'=[0,1)\times\{1\},\quad D'=(0,1]\times\{0\},\quad E'=\{\textstyle{\frac{1}{2}}\times0,\textstyle{\frac{1}{2}}\times1\}.
$$ 

\hs

\noi\textbf{Exercise 17.19 p. 102 of the book.} If $A\subset X$, we define the \textbf{boundary} of $A$ by the equation
$$
\operatorname{Bd}A=\overline{A}\cap(\overline{X-A})
$$
\begin{enumerate}
    \item[(a)] Show that $\operatorname{Int} A$ and $\operatorname{Bd} A$ are disjoint, and $\overline{A}=\operatorname{Int}A\cup\operatorname{Bd}A$.
    \item[(b)] Show that $\operatorname{Bd}A=\varnothing\Leftrightarrow A$ is both open and closed.
    \item[(c)] Show that $U$ is open $\Leftrightarrow\operatorname{Bd} U=\overline{U}-U$.
    \item[(d)] If $U$ is open, is it true that $U=\operatorname{Int}(\,\overline{U}\,)$ ? Justify your answer.
\end{enumerate}

\noi\textbf{Solution.} In this Solution we use the following notation: $A^c:=X\sm A,A^-:=$ closure of $A$, $A^o:=$ interior of $A$, $A^b:=$ boundary of $A$. Note that $A^b$ is closed, that $A^{c\,b}=A^b$, and that $A^b=A^-\cap A^{c-}=A^-\cap A^{o\,c}=A^-\sm A^o$. In particular, the equality 
$$
A^b=A^-\sm A^o
$$ 
implies (a) and (b). 

\noi(c) We must show $U^o=U\Leftrightarrow U^-\sm U^o=U^-\sm U$, which is clear. 

\noi(d) The answer is no, a counterexample being give by $\R\sm\{0\}$ in $\R$. 

\hs

\noi\textbf{Exercise 17.20 p. 102 of the book.} Find the boundary and the interior of each of the following subsets of $\mathbb{R}^{2}$:
\begin{enumerate}
    \item[(a)] $A=\{x\times y\mid y=0\}$
    \item[(b)] $B=\{x\times y\mid x>0\text{ and }y\ne0\}$
    \item[(c)] $C=A\cup B$
    \item[(d)] $D=\{x\times y\mid x\text{ is rational}\}$
    \item[(e)] $E=\{x\times y\mid 0<x^2-y^2\le1\}$
    \item[(f)] $F=\{x\times y\mid x\ne0\text{ and }y\leq1/x\}$. 
\end{enumerate}

\noi\textbf{Solution.} We use the same notations as in Exercise 17.19. 

\noi(a) $A^-=A$ ($A$ is closed), $A^o=\varnothing,\ A^b=A$. 

\noi(b) $B^-=\{x\times y\mid x\ge0,\ B^o=B$ ($B$ is open), $B^b=\{x\times y\mid x=0\}\cup\{x\times y\mid x\ge0\text{ and }y=0\}$. 

\noi(c) $C^-=\{x\times y\mid x\ge0\},\ C^o=\{x\times y\mid x>0\},\ C^b=\{x\times y\mid xy=0\text{ and }x\le0\}$. 

\noi(d) $D^-=\R^2,\ D^o=\varnothing,\ D^b=\R^2$. 

\noi(e) $E^-=\{x\times y\mid 0\le x^2-y^2\le1\},\ E^o=\{x\times y\mid 0<x^2-y^2<1\},\ E^b=\{x\times y\mid x^2-y^2\in\{0,1\}\}$. 

\noi(f) $F^-=\{x\times y\mid(x\neq 0\text{ and }y\leq1/x)\text{ or }x=0\},\ F^o=\{x\times y\mid x\neq 0\text{ and }y<1/x\},\\{}\hskip16pt F^b=\{x\times y\mid(x\neq 0\text{ and }y=1/x)\text{ or }x=0\}$. 

\hs 

\noi\textbf{Theorem 18.1 p. 104 of the book.} Recall the statement:

\begin{thm}[Theorem 18.1 of the book] 
Let $X$ and $Y$ be topological spaces; let $f: X \to Y$. Then the following are equivalent:
\begin{itemize}
    \item[$(1)$] $f$ is continuous.
    \item[$(2)$] For every subset $A$ of $X$, one has $f(\overline{A}) \subset \overline{f(A)}$.
    \item[$(3)$] For every closed set $B$ of $Y$, the set $f^{-1}(B)$ is closed in $X$.
    \item[$(4)$] For each $x \in X$ and each neighborhood $V$ of $f(x)$, there is a neighborhood $U$ of $x$ such that $f(U) \subset V$.
\end{itemize}
If the condition in $(4)$ holds for the point $x$ of $X$, we say that $f$ is \textbf{continuous at the point} $x$.
\end{thm} % variant https://docs.google.com/document/d/1I1JMPqDsBb9GX9Z7pExBEbHFRrgRjV5s89uDdqfRX3E/edit?tab=t.0

The proof that (4) implies (1) is written as follows: ``Let $V$ be an open set of $Y$; let $x$ be a point of $f^{-1}(V)$. Then $f(x) \in V$, so that by hypothesis there is a neighborhood $U_x$ of $x$ such that $f(U_x) \subset V$. Then $U_x \subset f^{-1}(V)$. It follows that $f^{-1}(V)$ can be written as the union of the open sets $U_x$, so that it is open.'' One can avoid using the Axiom of Choice by the following wording: $f^{-1}(V)$ can be written as the union of the open sets $U$ such that $x\in U\subset f^{-1}(V)$, so that it is open. (A similar comment was made after Lemma~\ref{L13.1}.) 

\hs

\noi\textbf{A corollary to Theorem 18.1 p. 104 of the book.} 

\begin{cor}\label{C18.1a}
Let $X$ and $X'$ be topological spaces, let $\B$ and $\B'$ be respective basis for the topology of $X$ and $X'$, let $f:X\to X'$ be a map, and $x$ a point of $X$. Then $f$ is continuous at $x$ if and only if, for all basic neighborhood $B'\in\B'$ of $f(x)$ there is a basic neighborhood $B\in\B$ of $x$ such that $f(B)\subset B'$. 
\end{cor} 

\begin{proof}
Let $f$ be continuous at $x$, and let $B'$ be a basic neighborhood of $f(x)$. Then there is a neighborhood $U$ of $x$ such that $f(U)\subset B'$, and any basic neighborhood $B$ of $x$ contained in $U$ will satisfy $f(B)\subset B'$. Conversely, given a neighborhood $U'$ of $f(x)$, pick a basic neighborhood $B'$ of $f(x)$ contained in $U'$, and note that there is a basic neighborhood $B$ of $x$ such that $f(B)\subset B'\subset U'$.  
\end{proof}

\smallskip\hrule\medskip 

\noi\textbf{Exercise 18.7 p. 111 of the book.} (a) Suppose that $f: \mathbb{R} \rightarrow \mathbb{R}$ is ``continuous from the right,'' that is,
\[
\lim _{x \rightarrow a^{+}} f(x)=f(a)
\]
for each $a \in \mathbb{R}$. Show that $f$ is continuous when considered as a function from $\mathbb{R}_{\ell}$ to $\mathbb{R}$.

\noi(b) Can you conjecture what functions $f: \mathbb{R} \rightarrow \mathbb{R}$ are continuous when considered as maps from $\mathbb{R}$ to $\mathbb{R}_{\ell}$ ? As maps from $\mathbb{R}_{\ell}$ to $\mathbb{R}_{\ell}$ ? We shall return to this question in Chapter 3.

\noi\textbf{Solution.} (a) Assume $f(x)\in(a,b)$ for some $x,a,b\in\R$. It suffices to show that there is a $d$ in $\R$ such that $x\in[x,d)$ and $[x,d)\subset f^{-1}((a,b))$, that is $x\in[x,d)$ and $f([x,d))\subset(a,b)$. We have 
\[
\lim _{y\rightarrow x^{+}}f(y)=f(x).
\] 
Set $\varepsilon:=\min(b-f(x),f(x)-a)$. There is a $\delta>0$ such that $y\in[x,x+\delta)$ implies $|f(y)-f(x)|<\varepsilon$, and thus $f(y)\in(a,b)$. This shows that $d:=x+\delta$ does the job. 

\hs 

\noi\textbf{Exercise 18.8 p. 111 of the book.} Let $Y$ be an ordered set in the order topology. Let $f, g: X \rightarrow Y$ be continuous.

\noindent (a) Show that the set $\{x \mid f(x) \leq g(x)\}$ is closed in $X$.

\noindent (b) Let $h:X\rightarrow Y$ be the function
$$
h(x)=\min \{f(x), g(x)\}
$$
Show that $h$ is continuous. [Hint: Use the pasting lemma.]

\noi\textbf{Solution.} I think that $X$ is implicitly assumed to be a topological space. 

\noi(a) Define $u:X\to Y^2$ by $u(x):=f(x)\times g(x)$. Note that $u$ is continuous. Set 
$$
B:=\{y\times z\in Y^2\mid y\le z\}.
$$ 
Then the set in the statement of (a) is $u^{-1}(B)$. Thus, it suffices to show that $B$ is closed. Set $U:=\{y\times z\in Y^2\mid y>z\}$. We must check that $U$ is open. Let $y\times z$ be in $U$. There are basic open sets $V,W\subset Y$ such $y\in V,z\in W$ and $y'>z'$ for all $y'\in V$ and all $z'\in W$. 

\hs

\noi\textbf{Exercise 18.13 p. 112 of the book.} Let $A\subset X$; let $f:A\to Y$ be continuous; let $Y$ be Hausdorff. Show that if $f$ may be extended to a continuous function $g:\overline A\to Y$, then $g$ is uniquely determined by $f$. 

\noi\textbf{Solution.} Let $h:\overline A\to Y$ be another continuous extension of $f$, assume by contradcition that $g(x_0)\ne h(x_0)$ for some $x_0$ in $\overline A$, and define $u:\overline A\to Y^2$ by $u(x):=g(x)\times h(x)$. In particular $u$ is continuous. By Exercise 17.13 p.~101 of the book, $U:=\{y\times y'\in Y^2\ |\ y\ne y'\}$ is open. Hence $u^{-1}(U)$ is open, and it is nonempty because it contains $x_0$, hence we have $a\in u^{-1}(U)$, that is, $g(a)\ne h(a)$ for some $a$ in $A$, contradicting the equalities $g(a)=f(a)=h(a)$. 

\hs

\noi\textbf{Exercise 19.1 p. 118 of the book.} Prove Theorem 19.2. 

\noi\textbf{Solution.} Recall the statement: 

\begin{thm}[Theorem 19.2 of the book]
Suppose the topology on each space $X_{\alpha}$ is given by a basis $\mc{B}_{\alpha}$. The collection of all sets of the form
\[
\prod_{\alpha\in J}B_\alpha
\]
where $B_{\alpha}\in\mc B_\alpha$ for each $\alpha$, will serve as a basis for the box topology on $\prod_{\alpha \in J} X_{\alpha}$. The collection of all sets of the same form, where $B_{\alpha} \in \mc{B}_{\alpha}$ for finitely many indices $\alpha$ and $B_{\alpha}=X_{\alpha}$ for all the remaining indices, will serve as a basis for the product topology $\prod_{\alpha \in J} X_{\alpha}$. 
\end{thm} 

\begin{proof} 
For each $\alpha\in J$ let $U_\alpha$ be open in $X_\alpha$, set $U:=\prod_{\alpha\in J}U_{\alpha}$, and let $x=(x_\alpha)_{\alpha\in J}$ be in $U$. By Lemma~\ref{L13.2b} p.~\pageref{L13.2b} it suffices to show that there is a set $B$ of the form $\prod_{\alpha\in J}B_\alpha$, where $B_\alpha\in\mc B_\alpha$ for each $\alpha$ such that $x\in B\subset U$, that is, $x_\alpha\in B_\alpha\subset U_\alpha$ for all $\alpha$. But this is clear. The case of the product topology is similar. 
\end{proof} 

\bigskip\hrule\medskip 

\noi\textbf{Exercise 19.2 p. 118 of the book.} Prove Theorem 19.3. 

\noi\textbf{Solution.} Recall the statement: 

\begin{thm}[Theorem 19.3 of the book]
Let $A_\alpha$ be a subspace of $X_\alpha$, for each $\alpha\in J$. Then $\prod A_\alpha$ is a subspace of $\prod X_\alpha$ if both products are given the box topology, or if both products are given the product topology.
\end{thm} 

\begin{proof} 
Consider first the box topology. %Set $X:=\prod X_\alpha$ and $A:=\prod A_\alpha$, 
Let $\T_b$ and $\T_s$ be respectively the box topology and the subspace topology on $\prod A_\alpha$ (when $X$ is given the box topology), let $\B_b$ be the basis of $\T_b$ consisting of all the products $\prod U_\alpha$ with $U_\alpha$ open in $A_\alpha$, that is, of all the products $\prod(A_\alpha\cap V_\alpha)$ with $V_\alpha$ open $X_\alpha$, and let $\B_s$ be the basis of $\T_s$ consisting of all the intersections $(\prod A_\alpha)\cap(\prod W_\alpha)$ with $W_\alpha$ open in $X_\alpha$. It suffices to show $\B_b=\B_s$, or, more concretely, $\prod(A_\alpha\cap V_\alpha)=(\prod A_\alpha)\cap(\prod V_\alpha)$. Let $x=(x_\alpha)$ be in $\prod X_\alpha$. Then 
$$
x\in\prod(A_\alpha\cap V_\alpha)\iff x_\alpha\in A_\alpha\cap V_\alpha\text{ for all }\alpha\iff x\in(\prod A_\alpha)\cap(\prod V_\alpha).
$$ 
The case of the product topology is similar. 
\end{proof} 

\smallskip\hrule\medskip 

\noi\textbf{Exercise 19.3 p. 118 of the book.} Prove Theorem 19.4. 

\noi\textbf{Solution.} Recall the statement: 

\begin{thm}[Theorem 19.4 of the book]
If each space $X_{\alpha}$ is Hausdorff space, then $\prod X_\alpha$ is a Hausdorff space in both the box and product topologies.
\end{thm} 

\begin{proof} 
Let $x$ and $y$ be distinct points of $\prod X_\alpha$, let $\alpha$ satisfy $x_\alpha\ne y_\alpha$, and let $U_\alpha$ and $V_\alpha$ be two disjoint open subsets of $X_\alpha$ such that $x_\alpha\in U_\alpha$ and $y_\alpha\in v_\alpha$. For $\beta\ne\alpha$ set $U_\beta=V_\beta=X_\beta$. Then $\prod U_\gamma$ and $\prod V_\gamma$ are two disjoint open subsets of $\prod X_\gamma$ (in both topologies) such that $x\in\prod U_\gamma$ and $y\in\prod V_\gamma$. 
\end{proof} 

\smallskip\hrule\medskip 

\noi\textbf{Exercise 19.6 p. 118 of the book.} Let $\mathbf{x}_1, \mathbf{x}_2, \ldots$ be a sequence of the points of the product space $\prod X_{\alpha}$. Show that this sequence converges to the point $\mathbf{x}$ if and only if the sequence $\pi_{\alpha}\left(\mathbf{x}_1\right), \pi_{\alpha}\left(\mathbf{x}_2\right), \ldots$ converges to $\pi_{\alpha}(\mathbf{x})$ for each $\alpha$. Is this fact true if one uses the box topology instead of the product topology? 

\noi\textbf{Solution.} If $x_1,x_2,\ldots$ converges in either topology, then $x_{1\alpha},x_{2\alpha},\ldots$ converges for all $\alpha$ because $\pi_\alpha$ is continuous. If $x_{1\alpha},x_{2\alpha},\ldots$ converges for all $\alpha$, then $x_1,x_2,\ldots$ converges in the product topology. Indeed, for all $\alpha$ let $x_\alpha$ be a limit of $x_{1\alpha},x_{2\alpha},\ldots$, and let $U=\prod U_\alpha$ be a neighborhood of $(x_\alpha)_{\alpha\in J}$. Since there is a finite subset $F$ of $J$ such that $U_\alpha=X_\alpha$ for $\alpha$ not in $J$, we see that $x_n$ will be in $U$ for all $n$ large enough. 

\bigskip\hrule\medskip 

\noi\textbf{Exercise 19.7 p. 118 of the book.} Let $\mathbb{R}^{\infty}$ be the subset of $\mathbb{R}^{\omega}$ consisting of all sequences that are ``eventually zero,'' that is, all sequences $\left(x_1, x_2, \ldots\right)$ such that $x_{i} \neq 0$ for only finitely many values of $i$. What is the closure of $\mathbb{R}^{\infty}$ in $\mathbb{R}^{\omega}$ in the box and product topologies? Justify your answer. 

\noi\textbf{Solution.} The subset $\R^\infty$ is closed in $\R^\omega$ equipped with the box topology. Indeed, let $x$ be in $\R^\omega\setminus\R^\infty$. Then there are infinitely many $i$ such that $x_i\ne0$. For each such $i$ set $U_i:=(-|x_i|/2,|x_i|/2)$. If $x_i=0$ set $U_i:=\R$. Finally set $U:=\prod U_i$. Then $U$ is a neighborhood of $x$ in the box topology, and $U$ is disjoint from $\R^\infty$. The closure of $\R^\infty$ in the product topology is $\R^\omega$. Indeed, let $x$ be in $\R^\omega$, and let $U=\prod U_i$ be a neighborhood of $x$. There is a finite subset $F$ of $\omega$ such that $U_i=\R$ for $i$ not in $F$. Define $y\in\R^\omega$ by $y_i=x_i$ if $i\in F$ and $y_i=0$ otherwise. Then $y$ is in $U\cap\R^\infty$. 

\bigskip\hrule\medskip 

\noi\textbf{Exercise 19.8 p. 118 of the book.} Given sequences $\left(a_1, a_2, \ldots\right)$ and $\left(b_1, b_2, \ldots\right)$ of real numbers with $a_i>0$ for all $i$, define $h: \mathbb{R}^{\omega} \rightarrow \mathbb{R}^{\omega}$ by the equation
    \[
    h((x_1,x_2,\ldots))=(a_1x_1+b_1,a_2x_2+b_2,\ldots)
    \]
    Show that if $\mathbb{R}^{\omega}$ is given the product topology, $h$ is a homeomorphism of $\mathbb{R}^{\omega}$ with itself. What happens if $\mathbb{R}^{\omega}$ is given the box topology? 

\noi\textbf{Solution.} In fact $h$ is a homeomorphism on both cases. The details are left to the reader. 

\bigskip\hrule\medskip 

\noi\textbf{Exercise 19.9 p. 118 of the book.} Show that the choice axiom is equivalent to the statement that for any indexed family $\left\{A_\alpha\right\}_{\alpha\in J}$ of nonempty sets, with $J\ne\varnothing$, the cartesian product
    \[
    \prod_{\alpha \in J} A_{\alpha}
    \]
    is not empty. 

\noi\textbf{Solution.} Let $(A_\alpha)_{\alpha\in J}$ be as above. The choice axiom says that $\prod A_\alpha\ne\varnothing$ if the $A_\alpha$ are \emph{disjoint}. The statement in the exercise clearly implies the choice axiom. To prove the converse, set $B_\alpha:=A_\alpha\times\{\alpha\}$ for all $\alpha$. Then the $B_\alpha$ are disjoint, and the choice axiom implies $\prod B_\alpha\ne\varnothing$. Let $b$ be in $\prod B_\alpha$. Then each $b_\alpha$ is of the form $(a_\alpha,\alpha)$ for some $a_\alpha$ in $A_\alpha$, and $(a_\alpha)_{\alpha\in J}$ is in $\prod A_\alpha$. 

\bigskip\hrule\medskip 

\noi\textbf{Exercise 19.10 p. 118 of the book.} Let $A$ be a set; let $\left\{X_{\alpha}\right\}_{\alpha \in J}$ be an indexed family of spaces; and let $\left\{f_{\alpha}\right\}_{\alpha \in J}$ be an indexed family of functions $f_{\alpha}: A \rightarrow X_{\alpha}$.
    \begin{enumerate}[(a)]
        \item Show there is a unique coarsest topology $\mathcal{T}$ on $A$ relative to which each of the functions $f_\alpha$ is continuous.
        \item Let
        \[
        \mc S_\beta= \left\{f_{\beta}^{-1}\left(U_{\beta}\right)\mid U_{\beta}\text{ is open in } X_\beta\right\}
        \]
        and let $\mc S=\bigcup\mc S_\beta$. Show that $s$ is a subbasis for $\mathcal{T}$.
        \item Show that a map $g:Y\rightarrow A$ is continuous relative to $\mathcal{T}$ if and only if each map $f_\alpha\circ g$ is continuous.
        \item Let $f:A\rightarrow\prod X_\alpha$ be defined by the equation
        \[
        f(a) = \left(f_\alpha(a)\right)_{\alpha \in J}
        \]
        let $Z$ denote the subspace $f(A)$ of the product space $\prod X_\alpha$. Show that the image under $f$ of each element of $\mathcal{T}$ is an open set of $Z$.
    \end{enumerate} 

\noi\textbf{Solution.} We define $\T$ as the topology generated by $\mc S$. Then (a) and (b) are clear. To prove (c), it suffices to show that $g^{-1}(f_\alpha^{-1}(U_\alpha))$ is open whenever $U_\alpha$ is open in $X_\alpha$. But we have $g^{-1}(f_\alpha^{-1}(U_\alpha))=(f_\alpha\circ g)^{-1}(U_\alpha)$, which is open because $f_\alpha\circ g$ is continuous. Here is a counterexample to (d). Set $A=J=\{1\},X_1=\{1\}$ and equip $X_1$ with the indiscrete topology. 

\bigskip\hrule\medskip 

\noi\textbf{About the definitions p. 119 of the book.} 

\noi Corollary \ref{C18.1a} p. \pageref{C18.1a} above was a first corollary to Theorem 18.1 p. 104 of the book. Here is a second one: 

\begin{cor}\label{C18.1b}
Let $X$ and $X'$ be metric spaces, let $f:X\to X'$ be a map, and $x$ a point of $X$. Then $f$ is continuous at $x$ if and only if, for all positive $\eps$ there is a positive $\delta$ such that $f(B(x,\delta))\subset B(f(x),\eps)$. 
\end{cor} 

\begin{proof}
This follows immediately from Corollary \ref{C18.1a}. 
\end{proof} 

If the convention of denoting the couple $(x,y)$ by $x\times y$ was strictly applied, we should write $d(x\times y)$ instead of $d(x,y)$ for the distance between $x$ and $y$. 

\bigskip\hrule\medskip 

\noi\textbf{Exercise 20.1 p. 126 of the book.} 
    \begin{enumerate}[(a)]
        \item In $\mathbb{R}^{n}$, define
        \[
        d'(\mathbf{x}, \mathbf{y})=\left|x_1-y_1\right|+\cdots+\left|x_{n}-y_{n}\right|.
        \]
        Show that $d'$ is a metric that induces the usual topology of $\mathbb{R}^{n}$. Sketch the basis elements under $d'$ when $n=2$.

        \item More generally, given $p\geq1$, define
        \[
        d'(\mathbf{x}, \mathbf{y})=\left[\sum_{i=1}^{n}\left|x_{i}-y_{i}\right|^{p}\right]^{1 / p}
        \]
        for $\mathbf{x}, \mathbf{y} \in \mathbb{R}^{n}$. Assume that $d'$ is a metric. Show that it induces the usual topology on $\mathbb{R}^{n}$.
    \end{enumerate}

\noi\textbf{Solution.} (b) For $p\ge1$ and $x\in\R^n$ we set $|x|_p:=(|x_1|^p+\cdots+|x_n|^p)^{1/p}$ and $|x|_\infty:=\max_i|x_i|$. For $x,y\in\R^n$ and $1\le p\le\infty$ we set $d_p(x,y):=|y=x|_p$. It was shown in the proof of Theorem 20.3 p.~123 of the book that $d_\infty$ is a metric, and that it induces the product topology on $\R^n$. We assume that $d_p$ is a metric for $1\le p<\infty$, and we must prove that it also induces the product topology on $\R^n$. We claim 
\begin{equation}\label{E20.1}
1\le\frac{d_p(x,y)}{d_\infty(x,y)}\le n^{1/p}=:a
\end{equation} 
for all $x\ne y$ in $\R^n$. We first show that this implies that $d_p$ induces the product topology on $\R^n$. We abbreviate $B_{d_p}$ by $B_p$. The above display implies 
$$
d_\infty(x,y)\le d_p(x,y)\le ad_\infty(x,y), 
$$ 
hence $B_p(x,r)\subset B_\infty(x,r)\subset B_p(x,ar)$, hence $d_p$ induces the product topology on $\R^n$. We prove \eqref{E20.1}. It suffices to show 
$$
1\le\frac{|x|_p^p}{|x|_\infty^p}\le n
$$  
for all $x\ne0$. We can assume $0\le x_1\le\cdots\le x_n=1$. We must show 
$$
1\le\sum_ix_i^p\le n,
$$ 
which is clear. 

Even if this is not requested, we show that $d_p$ is a metric for $p>1$. We will be content to prove the triangle inequality, the other properties being easy. The argument below is almost a copy-and-paste of the post \url{https://math.stackexchange.com/a/2283804/660} by Felix Benning. 

Let $p,q>1$. We have 
$$
\frac1p+\frac1q=1\iff p+q=pq\iff pq-(p+q)+1=1\iff(p-1)(q-1)=1
$$ 
$$
\iff p-1=\frac1{q-1}\iff p\,(q-1)=q\iff p-\frac pq=1.
$$ 
Young's Inequality: $xy\le\dfrac{x^p}p+\dfrac{y^q}q$ for $x,y\ge0$. 

\noi Proof. We can assume $x,y>0$. Set $f(x)=\dfrac{x^p}p+\dfrac{y^q}q-xy$. It suffices to show $f(x)\ge0$ for $x>0$. We have 
$$
f'(x)=x^{p-1}-y,\quad f''(x)=(p-1)\,x^{p-2}>0,\quad f'(x)=0\iff x=y^{\frac1{p-1}},
$$
$$
f'(x)=0\iff x=y^{q-1},\quad f(y^{q-1})=\frac{y^{p(q-1)}}p+\frac{y^q}q-y^q=\left(\frac1p+\frac1q-1\right)y^q=0
$$ 
We see that $f'$ is increasing and vanishes at $y^{q-1}$, hence $f'$ is negative on $(0,y^{q-1})$ and positive on $(y^{q-1},\infty)$, hence $f$ is decreasing on $(0,y^{q-1})$, zero at $y^{q-1}$, and increasing on $(y^{q-1},\infty)$. This implies $f(x)\ge0$ for $x>0$, as required. $\square$ 

\noi Rewrite Young's Inequality as $uv\le\dfrac{u^p}p+\dfrac{v^q}q$ for $u,v\ge0$.
 
\noi Hölder's Inequality: $\sum|x_i\,y_i|\le|x|_p\,|y|_q$, for $x_i,y_i\in\mathbb R,\ i=1,\ldots,n$.

\noi Proof. Setting $u_i=\dfrac{|x_i|}{|x|_p}$ and $v_i=\dfrac{|y_i|}{|y|_q}$ and using Young's inequality, we get 
\begin{align*}
&&\frac{|x_i\,y_i|}{|x|_p\,|y|_q}&\le\frac{|x_i|^p}{p\,|x|_p^p}+\frac{|y_i|^q}{q\,|y|_q^q}\\[8pt] 
\implies&&\sum\,\frac{|x_i\,y_i|}{|x|_p\,|y|_q}&\le\sum\,\frac{|x_i|^p}{p\,|x|_p^p}+\sum\,\frac{|y_i|^q}{q\,|y|_q^q}\\[8pt] 
\implies&&\frac{\sum x_i\,y_i}{|x|_p\,|y|_q}&\le\frac1p+\frac1q=1.\ \square
\end{align*} 

\noi We prove the triangle inequality $|x+y|_p\le|x|_p+|y|_p$. Setting $q:=\dfrac p{p-1}$ we get 
\begin{align*}
|x+y|_p^p&=\sum\,|x_i\,+y_i|^p\le\sum\,|x_i+y_i|^{p-1}\,|x_i|+\sum\,|x_i+y_i|^{p-1}\,|y_i|\\[8pt] 
&\le\left(\sum\,|x_i+y_i|^{(p-1)q}\right)^{\frac1q}\left(\sum\,|x_i|^p\right)^{\frac1p}+\left(\sum\,|x_i+y_i|^{(p-1)q}\right)^{\frac1q}\left(\sum\,|y_i|^p\right)^{\frac1p}\\[8pt] 
&=\left(\sum\,|x_i+y_i|^p\right)^{\frac1p\frac pq}\left(|x|_p+|y|_p\right)\\[8pt] 
&=\left(|x+y|_p\right)^{\frac pq}\,\left(|x|_p+|y|_p\right),
\end{align*} 
hence 
$$
|x+y|_p=|x+y|_p^{p-\frac pq}\le|x|_p+|y|_p.
$$ \smallskip\hrule\bigskip

\noi\textbf{Exercise 20.2 p. 126 of the book.} Show that $\mathbb{R} \times \mathbb{R}$ in the dictionary order topology is metrizable. 

\noi\textbf{Solution.} Define $d(x_1\times x_2,y_1\times y_2)$ by decreeing $d(t\times x_2,t\times y_2)=\min(|y_2-x_2|,1)$ and $d(x_1\times x_2,y_1\times y_2)=2$ if $x_1\ne y_1$. 

\bigskip\bigskip\hrule\bigskip 

\noi\textbf{Exercise 20.3 p. 126 of the book.} Let $X$ be a metric space with metric $d$.
    \begin{enumerate}[(a)]
        \item Show that $d: X\times X\to\R$ is continuous.
        \item Let $X'$ denote a space having the same underlying set as $X$. Show that if $d: X' \times X' \rightarrow \mathbb{R}$ is continuous, then the topology of $X'$ is finer than the topology of $X$.
    \end{enumerate}

    One can summarize the result of this exercise as follows: If $X$ has a metric $d$, then the topology induced by $d$ is the coarsest topology relative to which the function $d$ is continuous.  

\noi\textbf{Solution.} (a) Let $a,b\in\R$ with $a<b$. It suffices to show that $d^{-1}((a,b))$ is open. Let $x_0\times y_0\in X^2$ set $c:=d(x_0,y_0)$ and assume 
\begin{equation}\label{E20.3a}
a<c<b.
\end{equation}
It suffices to show that there is a positive $\eps$ such that $d(x_0,x),d(y_0,y)<\eps$ implies 
\begin{equation}\label{E20.3b}
a<d(x,y)<b.
\end{equation}
Let $\delta$ be positive. We have 
$$
d(x,y)\le d(x,x_0)+d(x_0,y_0)+d(y_0,y)\le c+2\delta.
$$ 
A similar computation shows $c\le d(x,y)+2\delta$, that is, $c-2\delta\le d(x,y)$, and thus 
\begin{equation}\label{E20.3c}
c-2\delta\le d(x,y)\le c+2\delta.
\end{equation} 
In view of \eqref{E20.3a}, \eqref{E20.3b} and \eqref{E20.3c}, it suffices to check that, for $\delta$ small enough, we have $a<c-2\delta$ and $c+2\delta<b$. Clearly $\delta=\min(\frac{c-a}2,\frac{b-c}2)$ does the job, and we can set $\eps:=\delta$. 

\noi(b) Let $x$ be in $X$ and $r$ be positive. It suffices to show that $B(x,r)$ is open in $X'$. Note that $U:=d^{-1}((-1,r))$ is open in $X'$. Let $p_2:X^2\to X$ be the second projection. We have 
$$
B(x,r)=p_2\big(U\cap(\{x\}\times X)\big)
$$ 
Let $y$ be in $B(x,r)$. It suffices to find an open subset $W$ of $X'$ such that $y\in W\subset B(x,r)$. By definition of the product topology, there are open subsets $V,W$ of $X'$ such that $x\in V,y\in W$ and $V\times W\subset U$, and we get $y\in W\subset B(x,r)$, as required.  

\bigskip\hrule\bigskip

\noi\textbf{Exercise 20.4 p. 127 of the book.} Consider the product, uniform, and box topologies on $\mathbb{R}^{\omega}$. 

    \begin{enumerate}[(a)]
        \item In which topologies are the following functions from $\mathbb{R}$ to $\mathbb{R}^{\omega}$ continuous?
        \begin{align*}
            f(t) &= (t, 2t, 3t, \ldots) \\
            g(t) &= (t, t, t, \ldots) \\
            h(t) &= \left(t, \frac12t, \frac13t, \ldots\right)
        \end{align*}
        \item In which topologies do the following sequences converge?
        \begin{align*}
            \mathbf{w}_1 &= (1,1,1,1, \ldots), & \mathbf{x}_1 &= (1,1,1,1, \ldots) \\
            \mathbf{w}_2 &= (0,2,2,2, \ldots), & \mathbf{x}_2 &= \left(0, \frac12, \frac12, \frac12, \ldots\right) \\
            \mathbf{w}_3 &= (0,0,3,3, \ldots), & \mathbf{x}_3 &= \left(0,0, \frac13, \frac13 \ldots\right) \\
            \ldots & \ldots \\
            \mathbf{y}_1 &= (1,0,0,0, \ldots), & \mathbf{z}_1 &= (1,1,0,0, \ldots) \\
            \mathbf{y}_2 &= \left(\frac12, \frac12, 0,0, \ldots\right), & \mathbf{z}_2 &= \left(\frac12, \frac12, 0,0, \ldots\right) \\
            \mathbf{y}_3 &= \left(\frac13, \frac13, \frac13, 0, \ldots\right), & \mathbf{z}_3 &= \left(\frac13, \frac13, 0,0, \ldots\right) \\
            \ldots & \ldots
        \end{align*}
    \end{enumerate}

\noi\textbf{Solution.} (a) We equip $\R^\omega$ with its natural $\R$-vector space structure, and observe that the three topologies considered are invariant by translation. Since $f,g$ and $h$ are linear, they are continuous if and only if they are continuous at 0. We can use Corollary~\ref{C18.1a} p.~\pageref{C18.1a}, and, for the uniform topology, Corollary~\ref{C18.1b} p.~\pageref{C18.1b}. It is easy to see that $f,g$ and $h$ are continuous for the product topology, and discontinuous for the box topology. So, let us endow $\R^\omega$ with the uniform topology, that is, the topology given by the metric $\overline\rho$. Let $\eps$ be positive and less than 1. We must decide of there is a positive $\delta$ such that $k((-\delta,\delta))\subset B_{\overline\rho}(0,\eps)$ for $k$ equal to $f,g$ or $h$. Since we have $\overline\rho(0,f(\frac\delta2))=1$, we see that $f$ is discontinuous. If $k$ equal to $g$ or $h$, and $|t|$ is less than $\frac\eps2$, we have $\overline\rho(0,g(t))<\eps$, and we can set $\delta:=\frac\eps2$. This shows that $g$ and $h$ are continuous. 

\noi(b) Assume that one of the sequences $u_n$ in the statement tends to $a\in\R^\omega$ in the topology $\T$. Since each projection $\R^\omega\to\R$ is continuous, we have $\lim_{n\to\infty}u_{ni}=a_i$ for all $i$. Since $\lim_{n\to\infty}u_{ni}=0$, this implies $a=0$. Conversely, it is easy to see that, if $\lim_{n\to\infty}u_{ni}=a_i$ for all $i$, then $u_n$ tends to 0 in the product topology. Clearly $z_n$ tends to 0 in each of the three topologies. 

We show that $y_n$ does not tend to 0 in the box topology. Set 
$$
U:=\prod_{i=1}^\infty\,(-i^{-2},i^{-2}). 
$$ 
Then $U$ is open in the box topology and contains 0. Then $y_n\in U$ for some $n$ would imply $|y_{ni}|<i^{-2}$ for all $i$, and thus $|y_{nn}|<n^{-2}$, that is $n^{-1}<n^{-2}$, contradiction. As a result, $y_n$ does not tend to 0 in the box topology. A similar argument applies to $w_n$ and $x_n$. 

It only remans to handle the uniform topology. In this case, $u_n$ tends to 0 if and only if $\overline\rho(0,u_n)$ does, that is, $u_n$ tends to 0 if and only if there is a positive integer $n_0$ and a sequence $a_1,a_2,\ldots$ of positive numbers such that $n\ge n_0$ and $i\in\omega$ imply $|u_{ni}|\le a_n$. In particular $w_n$ does not tend to 0, but the other sequences do. 

\bigskip\hrule\bigskip

\noi\textbf{Exercise 20.5 p. 127 of the book.} Let $\mathbb{R}^{\infty}$ be the subset of $\mathbb{R}^{\omega}$ consisting of all sequences that are eventually zero. What is the closure of $\mathbb{R}^{\infty}$ in $\mathbb{R}^{\omega}$ in the uniform topology? Justify your answer.

\noi\textbf{Solution.} The closure in question is the set $C$ of all sequences tending to 0. Let us show that $C$ is closed. Let $x_1,x_2,\ldots$ be a sequence not tending to 0. There is a positive $\eps$ such that the set $\{n\in\omega\ |\ |x_n|>\eps\}$ is infinite, and $B_{\overline\rho}(x,\eps/2)\cap\R^\infty=\varnothing$, so $C$ is closed. Let $x$ be in $C$, let $\eps$ be positive, and lets us prove $B_{\overline\rho}(x,\eps)\cap\R^\infty\ne\varnothing$. There is a $n_0$ in $\omega$ such that $n>n_0$ implies $|x_n|<\eps/2$. Define $y\in\R^\infty$ by $y_n=x_n$ if $n\le n_0$ and $y_n=0$ if $n>n_0$. Then $y$ is in $B_{\overline\rho}(x,\eps)$. 

\bigskip\bigskip\hrule\bigskip

\noi\textbf{Exercise 20.6 p. 127 of the book.} Let $\overline{\rho}$ be the uniform metric on $\mathbb{R}^{\omega}$. Given $\mathbf{x} = \left(x_1, x_2, \ldots\right) \in \mathbb{R}^{\omega}$ and given $0 < \eps < 1$, let
    \[
    U(\mathbf{x}, \eps) = \left(x_1 - \eps, x_1 + \eps\right) \times \cdots \times \left(x_{n} - \eps, x_{n} + \eps\right) \times \cdots
    \]
    \begin{enumerate}[(a)]
        \item Show that $U(\mathbf{x},\eps)$ is not equal to the $\eps$-ball $B_{\overline{\rho}}(\mathbf{x}, \eps)$.
        \item Show that $U(\mathbf{x},\eps)$ is not even open in the uniform topology.
        \item Show that
        \[
        B_{\overline{\rho}}(\mathbf{x},\eps)=\bigcup_{\delta < \eps}\,U(\mathbf{x}, \delta).
        \]
    \end{enumerate}

\noi\textbf{Solution.} (a) Define $y\in\R^\omega$ by $y_n:=x+\frac{n-1}n\,\eps$. Then $y$ is in $U(x,\eps)$ but not in $B_{\overline{\rho}}(x,\eps)$. 

\noi(b) We can assume $x=0$. Again, define $y\in\R^\omega$ by $y_n:=\frac{n-1}n\,\eps$. Then $y$ is in $U(0,\eps)$. Let $\delta$ be positive, and set $z_n:=y_n+\frac\delta2=\frac{n-1}n\eps+\frac\delta2$. Then $z$ is in $B_{\overline{\rho}}(y,\delta)$ but not in $U(0,\eps)$. 

\noi(c) We can again assume $x=0$.  The inclusion $\bigcup_{\delta<\eps}\,U(0,\delta)\subset B_{\overline{\rho}}(0,\eps)$ is clear. To prove the converse inclusion, let $y$ be in $B_{\overline\rho}(0,\eps)$, and let $\alpha$ satisfy $\overline\rho(0,y)<\alpha<\eps$. There is an $n_0$ such that $|y_n|<\alpha$ for $n>n_0$, and a $\delta$ such that $\alpha<\delta<\eps$ and $|y_n|<\delta$ for all $n$, that is, $y\in U(0,\delta)$. 

\bigskip\bigskip\hrule\bigskip

\noi\textbf{Exercise 20.7 p. 127 of the book.} Consider the map $h: \mathbb{R}^{\omega} \rightarrow \mathbb{R}^{\omega}$ defined in Exercise 8 of $\S 19$; give $\mathbb{R}^{\omega}$ the uniform topology. Under what conditions on the numbers $a_{i}$ and $b_{i}$ is $h$ continuous? a homeomorphism? 

\noi\textbf{Solution.} Recall that $h$ is defined by $h(x_1,x_2,\ldots)=(a_1x_1+b_1,a_2x_2+b_2, \ldots)$ with $a_i>0$ for all $i$. 

We claim that $h$ is continuous if and only if the $a_i$ are bounded, and that $h$ is a homeomorphism if and only if the $a_i$ and the $a_i^{-1}$ are bounded. 

Set $h'(x_1,x_2,\ldots)=(a_1x_1,a_2x_2,\ldots)$. It is easy to see that $h$ is continuous if and only if $h'$ is continuous if and only if $h'$ is continuous at 0, and that $h$ is a homeomorphism if and only if $h'$ and $h'^{-1}$ are continuous at 0. In other words, we can assume $b_i=0$ for all $i$. We shall use Corollary~\ref{C18.1b} p.~\pageref{C18.1b} above. 

If $a_i<c$ for all $i$ and $\eps$ is positive and less than 1, then $h(B(0,\eps/c))\subset B(0,\eps)$. 

Assume that the $a_i$ are not bounded, let $\eps$ be as above, and let $\delta$ be positive and less than 1. Then $(\frac\delta2,\frac\delta2,\ldots)\in B(0,\delta)$ but $h(\frac\delta2, \frac\delta2,\ldots)\notin B(0,\eps)$; in particular $h(B(0,\delta))\not\subset B(0,\eps)$. 

Finally, not that $h^{-1}(x_1,x_1,\ldots)=(a_1^{-1}x_1,a_2^{-1}x_2,\ldots)$. This proves the claim. 

\bigskip\bigskip\hrule\bigskip

\noi\textbf{Exercise 20.8 p. 127 of the book.} Let $X$ be the subset of $\mathbb{R}^{\omega}$ consisting of all sequences $\mathbf{x}$ such that $\sum x_{i}^{2}$ converges. Then the formula
    \begin{equation*}
    d(\mathbf{x}, \mathbf{y})=\left[\sum_{i=1}^{\infty}\left(x_{i}-y_{i}\right)^{2}\right]^{1 / 2}
    \end{equation*}
    defines a metric on $X$. (See Exercise 10.) On $X$ we have the three topologies it inherits from the box, uniform, and product topologies on $\mathbb{R}^{n}$. We have also the topology given by the metric $d$, which we call the $\ell^{2}$-topology. (Read "little ell two.")
    \begin{enumerate}[(a)]
        \item Show that on $X$, we have the inclusions
        \begin{equation*}
        \text{box topology} \supset \ell^{2}\text{-topology} \supset \text{uniform topology}.
        \end{equation*}
        \item The set $\mathbb{R}^{\infty}$ of all sequences that are eventually zero is contained in $X$. Show that the four topologies that $\mathbb{R}^{\infty}$ inherits as a subspace of $X$ are all distinct.
        \item The set
        \begin{equation*}
        H=\prod_{n \in \mathbb{Z}_{+}}[0,1/n]
        \end{equation*}
        is contained in $X$; it is called the Hilbert cube. Compare the four topologies that $H$ inherits as a subspace of $X$.
    \end{enumerate} 
   
\noi\textbf{Solution.} \textbf{(a)} Let $I$ be the interval $(0,1)$. If $\eps_*=(\eps_1,\eps_2,\ldots)\in I^\omega$, set 
$$
B(\eps_*):=X\cap((-\eps_1,\eps_1)\times(-\eps_2,\eps_2)\times\cdots)=\{x\in X\ |\ |x_i|<\eps_i\ \forall\ i\}.
$$ 
If $\eps\in I$, set $B_2(\eps):=B_d(0,\eps)$ and $B_u(\eps):=X\cap B_{\overline\rho}(0,\eps)$. An easy argument implies that it suffices to show: 

(A) for all $\eps\in I$ there is a $\delta_*\in I^\omega$ such that $B(\delta_*)\subset B_2(\eps)$, 

(B) for all $\eps\in I$ there is a $\delta\in I$ such that $B_2(\delta)\subset B_u(\eps)$. 

\noi To prove (A) (resp. (B)) set $\delta_n:=2^{-n/2}\eps$ (resp. $\delta=\eps$). 

\noi\textbf{(b)} Let us denote the box, $\ell^2$, uniform and product topologies on $X$ by $\T_b,\T_2,\T_u,\T_p$ respectively, and denote the respective topologies induced on $\R^\infty$ by $\T'_b,\T'_2,\T'_u,\T'_p$. By (a) we have $\T'_b\supset\T'_2\supset\T'_u\supset\T'_p$, and we must show $\T'_b\supsetneqq\T'_2\supsetneqq\T'_u\supsetneqq\T'_p$. It suffices to prove $\T'_b\ne\T'_2\ne\T'_u\ne\T'_p$. In the notation of the solution to (a) set 
$$
C(\eps_*):=\R^\infty\cap B(\eps_*),\quad C_2(\eps):=\R^\infty\cap B_2(\eps),\quad C_u(\eps):=\R^\infty\cap B_u(\eps),
$$ 
and, for $\eps\in I$ and $n\in\omega$ set $C(\eps,n):=\{x\in\R^\infty\ |\ |x_i|<\eps\ \forall\ i\le n\}$. Then the $C(\eps_*)$ and their translates form a basis of $\T'_b$, the $C_2(\eps)$ and their translates form a basis of $\T'_2$, the $C_u(\eps)$ and their translates form a basis of $\T'_u$, and the $C(\eps,n)$ and their translates form a basis of $\T'_p$. We show: 

\noi$\bullet\ \T'_b\ne\T'_2$: It suffices to check that $C_2(\delta)\not\subset C(1,\frac12,\frac13,\ldots)$ for all positive $\delta$. In fact, let $i\in\omega$ be larger than $\frac2\delta$, let $e_i$ be the $i$-th vector of the canonical basis of $\R^\infty$, and observe that $\frac\delta2e_i$ is in $C_2(\delta)$ but not in $C(1,\frac12,\frac13,\ldots)$. 

\noi$\bullet\ \T'_2\ne\T'_u$: It suffices to check that $C_u(\delta)\not\subset C_2(1)$ for all positive $\delta$. Indeed, let $n\in\omega$ be larger than $\frac4{\delta^2}$ and, in the above notation, set $x=\frac\delta2\sum_{i=1}^ne_i$, and observe that $x$ is in $C_u(\delta)$ but not in $C_2(1)$. 

\noi$\bullet\ \T'_u\ne\T'_p$: It suffices to check that $C(\delta,n)\not\subset C_u(1)$ for all positive $\delta$ and all $n\in\omega$. This follows from the fact that $e_{n+1}$ is in $C(\delta,n)$ but not in $C_u(1)$. % previous version https://docs.google.com/document/d/15xbQXm-dmAYxgy8cYzqEnGTp5pX6xzppT_k7_phvlUQ/edit?tab=t.0

\noi\textbf{(c)} Recall that we denote the box, $\ell^2$, uniform and product topologies on $X$ by $\T_b,\T_2,\T_u,\T_p$ respectively, and denote the respective topologies induced on $H$ by $\T''_b,\T''_2,\T''_u,\T''_p$. We claim $\T''_b\supsetneqq\T''_2=\T''_u=\T''_p$. In view of (a) and (b) it suffices to show $\T''_b\ne\T''_2=\T''_p$. For $x\in X$ define $t_x:X\to X$ by $t_x(y)_i:=x_i+y_i$, and set 
$$
D(x,\eps_*):=\R^\infty\cap t_x(C(\eps_*)),\quad D_2(x,\eps):=\R^\infty\cap t_x(C_2(\eps)),
$$ 
$$
D_u(x,\eps):=\R^\infty\cap t_x(C_u(\eps)),\quad D(x,\eps,n):=\R^\infty\cap t_x(C(\eps,n))
$$ 
for $x\in H,\eps_*\in I^\omega,\eps\in I$ and $n\in\omega$. Then the $D(x,\eps_*)$ form a basis of $\T''_b$, the $D_2(x,\eps)$ form a basis of $\T''_2$, the $D_u(x,\eps)$ form a basis of $\T''_u$, and the $D(x,\eps,n)$ form a basis of $\T''_p$. 

\noi$\bullet\ \T''_b\ne\T''_2$: It suffices to show that there is an $\eps_*\in I^\omega$ such that $D_2(0,\delta)\not\subset D(0,\eps_*)$ for all $\delta\in I$. We claim $D_2(0,\delta)\not\subset D(0,(2^{-1},2^{-2},2^{-3},\ldots))$ for all $\delta\in I$. To prove this, let $\delta$ be in $I$. Set $x_i:=2^{-i/2}$. We get $d(0,x)=1$ and $d(0,\frac\delta2\,x)= \frac\delta2$, so $\frac\delta2\,x\in D_2(0,\delta)$. We claim $\frac\delta2\,x\notin D(0,\eps_*)$. To prove this it suffices to show $\frac\delta2\,x_i\ge\eps_i$, that is $\frac\delta2\,2^{-i/2}\ge2^{-i}$, that is $2^{i/2}\ge\frac2\delta$, for $i$ large enough, which is clear. 

\noi$\bullet\ \T''_2=\T''_p$: Let $x$ be in $H$ and $\eps$ be in $I$. It suffices to show that there an $n\in\omega$ such that $D(x,\frac1n,n)\subset D_2(\eps)$. Let $n\in\omega$ and $y\in D(x,\frac1n,n)$ be arbitrary. Set $z_i:=y_i-x_i$ for all $i$. We want a condition on $n$ which implies $y\in D_2(\eps)$, that is, $\sum_{i=1}^\infty z_i^2<\eps^2$. We have 
$$
\sum_{i=1}^\infty z_i^2=\sum_{i=1}^nz_i^2+\sum_{i=n+1}^\infty z_i^2<n\,\frac1{n^2}+\sum_{i=n+1}^\infty\frac1{i^2}=\frac1n+\sum_{i=n+1}^\infty\frac1{i^2}=:f(n),
$$ 
and we can pick an $n$ such that $f(n)<\eps^2$. 

\bigskip\bigskip\hrule\bigskip

\noi\textbf{Exercise 20.9 p. 128 of the book.} Show that the euclidean metric $d$ on $\mathbb{R}^{n}$ is a metric, as follows: If $\mathbf{x}, \mathbf{y} \in \mathbb{R}^{n}$ and $c \in \mathbb{R}$, define
    \begin{align*}
    \mathbf{x}+\mathbf{y} & =\left(x_{1}+y_{1}, \ldots, x_{n}+y_{n}\right) \\
    c \mathbf{x} & =\left(c x_{1}, \ldots, c x_{n}\right) \\
    \mathbf{x} \cdot \mathbf{y} & =x_{1} y_{1}+\cdots+x_{n} y_{n}
    \end{align*}
    \begin{enumerate}[(a)]
        \item Show that $\mathbf{x} \cdot(\mathbf{y}+\mathbf{z})=(\mathbf{x} \cdot \mathbf{y})+(\mathbf{x} \cdot \mathbf{z})$.
        \item Show that $|\mathbf{x} \cdot \mathbf{y}| \leq\|\mathbf{x}\|\|\mathbf{y}\|$. [Hint: If $\mathbf{x}, \mathbf{y} \neq 0$, let $a=1 /\|\mathbf{x}\|$ and $b=1 /\|\mathbf{y}\|$, and use the fact that $\|a \mathbf{x} \pm b \mathbf{y}\| \geq 0$.]
        \item Show that $\|\mathbf{x}+\mathbf{y}\| \leq\|\mathbf{x}\|+\|\mathbf{y}\|$. [Hint: Compute $(\mathbf{x}+\mathbf{y}) \cdot(\mathbf{x}+\mathbf{y})$ and apply (b).]
        \item Verify that $d$ is a metric.
    \end{enumerate}

\noi\textbf{Solution.} Left to the reader. 

\bigskip\bigskip\hrule\bigskip

\noi\textbf{Exercise 20.10 p. 128 of the book.} Let $X$ denote the subset of $\mathbb{R}^{n}$ consisting of all sequences $\left(x_{1}, x_{2}, \ldots\right)$ such that $\sum x_{i}^{2}$ converges. (You may assume the standard facts about infinite series. In case they are not familiar to you, we shall give them in Exercise 11 of the next section.)
    \begin{enumerate}[(a)]
        \item Show that if $\mathbf{x},\mathbf{y}\in X$, then $\sum\left|x_i\,y_i\right|$ converges. [Hint: Use (b) of Exercise 9 to show that the partial sums are bounded.]
        \item Let $c \in \mathbb{R}$. Show that if $\mathbf{x}, \mathbf{y} \in X$, then so are $\mathbf{x}+\mathbf{y}$ and $c\mathbf{x}$.
        \item Show that
        \begin{equation*}
        d(\mathbf{x}, \mathbf{y})=\left[\sum_{i=1}^{\infty}\left(x_{i}-y_{i}\right)^{2}\right]^{1 / 2}
        \end{equation*}
        is a well-defined metric on $X$.
    \end{enumerate}

\noi\textbf{Solution.} Left to the reader. 

\bigskip\bigskip\hrule\bigskip

\noi\textbf{Exercise 20.11 p. 128 of the book.} Show that if $d$ is a metric for $X$, then
    \begin{equation*}
    d^{\prime}(x,y)=d(x,y)/(1+d(x,y))
    \end{equation*}
    is a bounded metric that gives the topology of $X$. [Hint: If $f(x)=x/(1+x)$ for $x>0$, use the mean-value theorem to show that $f(a+b)-f(b)\le f(a)$.]

\noi\textbf{Solution.} We prove the hint and leave the rest to the reader. We can assume $0<a\le b$. We have $f'(x)=\frac1{(1+x)^2}$. The mean-value theorem implies $f(a+b)-f(b)=af'(c)=\frac a{(1+c)^2}$ for some $c\in(b,a+b)$. It suffices to show $f'(c)\le\frac{f(a)}a$, or, equivalently, $\frac a{f(a)}\le\frac1{f'(c)}$. We have 
$$
\frac a{f(a)}=1+a\le1+b\le(1+c)^2=\frac1{f'(c)}\ .
$$ 

\end{document}
